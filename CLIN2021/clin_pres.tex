% \documentclass[aspectratio=169]{beamer} % handout removes dynamic effects
\documentclass[aspectratio=169,handout]{beamer}
\usecolortheme{spruce}
\usetheme{X}

\usepackage{soul}
\usepackage{tabularx}
\usepackage{stmaryrd}
\usepackage{proof}
\usepackage{bm}
\usepackage{booktabs}
\usepackage{multirow}
\usepackage{multicol}
\usepackage{emoji} % requires LuaLaTex compiler, is it worthy?
\usepackage[outline]{contour}
\usepackage{array}
\usepackage{colortbl}
\usepackage{amssymb}
\usepackage{xpatch}

\usepackage{naturaltableau}

\usepackage[backend=biber,style=authortitle,hyperref=false]{biblatex}
\bibliography{../clin-abstract/bibliography.bib}
%\AtEveryCitekey{\clearname{address}\clearfield{url}\clearfield{location}}
\AtEveryCitekey{\iffootnote{\smaller[2]\color{pred}}{}}
\renewcommand\thefootnote{\textcolor{pred}{\arabic{footnote}}}
\xapptobibmacro{cite}{\setunit{\nametitledelim}\printfield{year}}{}{}

% reset footnote counter for every frame
\AtBeginEnvironment{frame}{\setcounter{footnote}{0}}

\newif\ifshowall % use for working on single frame
% \showallfalse
\showalltrue

% cell coloring for heatmap
\newcommand{\hmc}[1]{\textbf{\cellcolor{black!#1}\ifnum #1>50\color{white}\fi{#1}}}

\newcommand{\bluetxt}[1]{\textcolor{blue}{#1}}
\definecolor{oranje}{rgb}{1.0, 0.49, 0.0}
\newcommand{\oranjetxt}[1]{\contour{blue}{\textcolor{oranje}{#1}}}
\newcommand{\transl}[1]{\text{\emph{#1}}}
\newcommand{\disj}{\,|\,}
\newcommand{\hide}[1]{}
\newcommand{\hlt}[1]{\textcolor{red}{#1}}

\newcommand{\depBI}[1]{#1^{\scriptscriptstyle\Box}\,}
\newcommand{\depBE}[1]{#1_{\scriptscriptstyle\Box}\,}
\newcommand{\depDI}[1]{#1^{\scriptscriptstyle\diamondsuit}\,}
\newcommand{\depDE}[1]{#1_{\scriptscriptstyle\diamondsuit}\,}

\newcommand{\type}[1]{\ensuremath\mathtt{#1}}
\newcommand{\np}{\type{np}}
\newcommand{\n}{\type{n}}
\newcommand{\pr}{\type{pr}}
\newcommand{\pp}{\type{pp}}
\newcommand{\ppart}{\type{part}}
\newcommand{\smain}{\type{s_{main}}}
\newcommand{\ssub}{\type{s_{sub}}}
\newcommand{\pron}{\type{pron}}
\newcommand{\s}{\type{s}}
\newcommand{\ahi}{\type{ahi}}
\newcommand{\ti}{\type{ti}}
\newcommand{\ww}{\type{ww}}
\newcommand{\bw}{\type{bw}}

\newcommand{\ensLP}{LangPro\,2$\times$2}
\newcommand{\li}{\!\multimap\!}
\newcommand{\term}[1]{#1}
\newcommand{\mill}{ILL${}_{\multimap,\diamondsuit,\Box}$}
\newcommand{\ps}{\vphantom{()}}
\usepackage{tikz}
% Tikz configuration/libraries/macros
\usetikzlibrary{arrows,matrix,graphs,shapes,decorations,automata,backgrounds,petri,positioning,calc,decorations.pathmorphing,external,shapes.multipart}
\usetikzlibrary{decorations.pathreplacing}
\tikzstyle{underbrace style}=[decorate,decoration={brace,raise=7mm,amplitude=5pt,mirror}]
\tikzstyle{underbrace text style}=[below, pos=.5, yshift=-8mm]

\newcommand{\blocktitle}[1]{\normalsize\textcolor{pred}{#1}}
\newcommand{\posimpl}{\overset{+}{\multimap}}
\newcommand{\negimpl}{\overset{-}{\multimap}}
\newcommand{\posat}[1]{\overset{+}{#1\proofspace}}
\newcommand{\negat}[1]{\overset{-}{#1\proofspace}}
\hypersetup{
    colorlinks=true,
    linkcolor=black,
    filecolor=magenta,      
    urlcolor=cyan,
}


\begin{document}

% \renewcommand\@makefntext[1]{%
%   \setstretch{1.0}
%   \iftoggle{brkText}
%     {\normalfont[\@thefnmark]\enspace #1}
%     {\mkbibsuperscript{\normalfont\@thefnmark}\enspace #1}%
%   \global\togglefalse{brkText}}


\title[]{Logic-based NLI for Dutch}
\author{
    Lasha Abzianidze \and 
    Konstantinos Kogkalidis
	}
\institute{
     Utrecht Institute of Linguistics OTS, Utrecht University}
\date[]{
09-07-2021 $@$ CLIN
}

{%
\setbeamertemplate{headline}{}
\frame{\titlepage}
}


\section{Overview}


%████████████████████████████████
\begin{frame}{Logic-based approach to NLI}

\begin{center}
\begin{tikzpicture}[node distance=4mm and 15mm,
tool/.style={color=white, rectangle, rounded corners=1mm, font=\scriptsize\bfseries, fill=black!70},
kb/.style={rectangle, font=\small, color=white, fill=black!90, draw=black},
data/.style={color=black!50!green, font=\footnotesize, minimum width=40mm},
toolspec/.style={color=white,font=\footnotesize,color=black!50!green,align=left,node distance=1mm and 1mm}
]
\node (raw) [data] 
{\alt<2->{\oranjetxt{SICK}-\emoji{flag-netherlands}\footcite{wijnholds-moortgat-2021-sick}}{Raw sentences}};
\node (parser)  [below=of raw, tool, minimum width=50mm] 
{\alt<2->{
\oranjetxt{Alpino \&
Neural Proof Nets}
}{Syntactic parser}};
\node (synt)  [below=of parser, data]  
{\alt<2->{\oranjetxt{$\lambda^{\multimap, \diamondsuit, \Box}$ terms}}{Syntactic structures}};
\node (rules) [below=of synt, tool] {
\alt<2->{\oranjetxt{Semantic morphism}}{
Conversion rules}};
\node (log)  [below=of rules, data]  
{\alt<2->{\oranjetxt{$\lambda^{\to}$ terms}}{Logical Forms}};
\node (wn)  [right=37mm of log, tool, minimum width=40mm] 
{
\footnotesize
\alt<2->{\oranjetxt{\textsmaller[1]{Open Dutch WordNet}}}{
Knowledge base}};
\node (tp)  [below right=7mm and 5mm of log, tool, minimum width=25mm] 
{
\alt<2->{\oranjetxt{LangPro}}{Theorem prover}};
\node (out)  [below=of tp, data] {
\alt<2->{\oranjetxt{\{N, E, C\}}}{
Relation label}};

\draw[->] (raw) -- (parser);
\draw[->] (parser) -- (synt);
\draw[->] (synt) -- (rules);
\draw[->] (rules) -- (log);
\draw[->] (log.south) -- (tp);
\draw[->] (wn.south) -- (tp);
\draw[->] (tp) -- (out);
\end{tikzpicture}
\end{center}

\end{frame}


\section{Syntax}

%\subsection{The linear $\lambda$-calculus}
\ifshowall
%████████████████████████████████
\begin{frame}{The linear $\lambda$-calculus}
    % 1. explain technicalities:
    %   a. types, give vocal examples of atomic vs complex, higher-order
    %   b. terms, how parsing is proving in the logic, and how logics are 1-1 with λ-calculi
    %   c. say that the actual parses uses an extended vocabulary of types & terms to capture dependency structure but ignore for now
    % 2. give motivations:
    %   a. these representations are already reminiscent of LLF which we use for inference
    %   b. explain that the interpretation domain is arbitrary, allowing a smooth transition to semantics of any kind
    % 3. introduce next slide -- if we get a proof we get a term, but how do we represent a proof?
    Words are assigned ILL \textbf{types}, inductively defined as:
    $
        \mathcal{T} := a \ | \ t_1 \li t_2 \ 
        \textcolor{gray}{| \ \Box^\alpha t \ | \ \diamondsuit^\delta t}
    $
    
    where
    \begin{itemize}
    \item $a$ an atom, from a finite set $\mathcal{A}$:\\ 
    \quad\small{
        {$\np$, $\smain$, $\ssub$, $\pron$, \dots}
        }
    \item $t_1 \li t_2$ a linear function that \textbf{consumes} $t_1$ to produce $t_2$\\
        \quad\small{
            $\np \li \smain$, $\np \li (\np \li \smain)$,
            $(\np \li \ssub) \li (\np \li \np)$, \dots
        }
    \end{itemize}\vfill
    
    \pause
    Syntactic derivations $\equiv$ \textbf{proofs} $\overset{chc}{\equiv}$ functional programs:\\
    \hfill
    $
    \tau := c^{t} \ |
    \ \left( s^{t_1\multimap t_2}~ u^{t_1}\right)^{t_2} \ |
    \ \left( \lambda x^{t_1}.s^{t_2} \right)^{t_1 \multimap t_2} \ 
    \textcolor{gray}{
        | \ \depBE{\alpha} \tau \ | \ \depBI{\alpha} \tau \ | \ \depDE{\delta} \tau \ | \ \depDI{\delta} \tau
    }
    $\hfill 
    
    \vfill\pause
    \center{\emph{Dit is een voorbeeld}}
    \vspace{-10pt}
    \[
    \synt{is}^{\np\li~\pron~\li~\smain}~
        \left(
            \synt{een}^{\n\li~\np}
            ~
            \synt{voorbeeld}^{\n}
        \right)
        ~
        \synt{dit}^{\pron}
    \]    
    
    
\end{frame}


%████████████████████████████████
\begin{frame}{Proof Nets}
    %   1. explain technicalities
    %       a) what is a formula unfolding, a proof frame, a proof net?
    %       b) why are proof nets a better alternative to other proof calculi?
    \alt<6->{
        \textbf{Proof net}\\
        a polarized forest (\textbf{proof frame}) and a bijection between \textcolor{cyan}{$+$} and \textcolor{red}{$-$} (axiom links)
    }{
        Types are \textbf{trees}, with nodes inductively \textbf{polarized}:
        \hfill \textcolor{cyan}{$+$} we \textbf{have} \hfill \alert{$-$} we \textbf{seek} \hfill ~
    }
    
    \vfill
    
    \pause
    \small\begin{center}
    \begin{tikzpicture}
        % words
        \node (a)                   at (0em, 0em)               {een};
        \node (very)                at (5em, 0em)               {heel};
        \node (easy)                at (10em, 0em)              {eenvoudig};
        \node (example)             at (15em, 0em)              {vorbeeld};
        
        % trees
        \node (a_tensor)            at (0em, 2em)               
            {\alt<3->{\textcolor{cyan}{$\li$}}{$\li$}};
        \node (a_n)                 at (-1em, 4em)              
            {\alt<4->{\textcolor{red}{\ps{$n$}}}{\ps{$n$}}};
        \node (a_np)                at (1em, 4em)               
            {\alt<4->{\textcolor{cyan}{\ps{$np$}}}{\ps{$np$}}};
        
        \node (very_tensor1)        at (5em, 2em)               
            {\alt<3->{\textcolor{cyan}{$\li$}}{$\li$}};
        \node (very_tensor2)        at (6.5em, 4em)             
            {\alt<4->{\textcolor{cyan}{$\li$}}{$\li$}}; 
        \node (very_cot)            at (3.5em, 4em)             
            {\alt<4->{\textcolor{red}{$\li$}}{$\li$}};
        \node (very_n1)             at (2.75em, 6em)            
            {\alt<5->{\textcolor{cyan}{\ps{$n$}}}{\ps{$n$}}};
        \node (very_n2)             at (4.25em, 6em)
            {\alt<5->{\textcolor{red}{\ps{$n$}}}{\ps{$n$}}};
        \node (very_n3)             at (5.75em, 6em)
            {\alt<5->{\textcolor{red}{\ps{$n$}}}{\ps{$n$}}};
        \node (very_n4)             at (7.25em, 6em)
            {\alt<5->{\textcolor{cyan}{\ps{$n$}}}{\ps{$n$}}};
        
        \node (easy_tensor)         at (10em, 2em)              
            {\alt<3->{\textcolor{cyan}{$\li$}}{$\li$}};
        \node (easy_n1)             at (9em, 4em)               
            {\alt<4->{\textcolor{red}{\ps{$n$}}}{\ps{$n$}}};
        \node (easy_n2)             at (11em, 4em)              
            {\alt<4->{\textcolor{cyan}{\ps{$n$}}}{\ps{$n$}}};
        
        \node (ex_n)                at (15em, 2em)
            {\alt<3->{\textcolor{cyan}{$n$}}{$n$}};
        
        % internal links -- static
        \draw (a_n)         --      (a_tensor)          --          (a_np);
        \draw (very_cot)    --      (very_tensor1)      --          (very_tensor2);
        \draw (very_n1)     --      (very_cot)          --          (very_n2);
        \draw (very_n3)     --      (very_tensor2)      --          (very_n4);
        \draw (easy_n1)     --      (easy_tensor)       --          (easy_n2);
        
        % axiom links 
        \visible<7->{
        \draw[->] (easy_n2)     --      ($(easy_n2) + (0em, 5em)$)      -|      (very_n2);
        \draw[->] (very_n1)     --      ($(very_n1) + (0em, 4em)$)      -|      (easy_n1);
        }
        \visible<8->{
        \draw[->] (ex_n)        --      ($(ex_n) + (0em, 6em)$)         -|      (very_n3);
        }
        \visible<9->{
        \draw[->] (very_n4)     --      ($(very_n4) + (0em, 5em)$)      -|      (a_n);
        \draw[->] (a_np)        --      ($(a_np) + (0em, 3em)$);
        }
    \end{tikzpicture}
    
    \[
        \visible<9->{\texttt{een}~(}
        \visible<7->{
        % \texttt{very}~\left(\lambda x.\texttt{easy}~x\right)}~
        \texttt{heel}~ \texttt{eenvoudig}}~
        \visible<8->{\texttt{vorbeeld}}
        \visible<9->{)}
    \]
    \end{center}
\end{frame}


%\subsection{Obtaining proof nets}
%████████████████████████████████
\begin{frame}
{Neural Proof Nets\footcite{npn}: from sentences to $\lambda$-terms}
    \begin{block}{\textbf{Supertagging}}
        From sentences to proof frames with \textit{seq2seq} transduction
    \end{block}
    \visible<10->{
    \begin{block}{\textbf{Proving}}
        From proof frames to axiom links with \textit{Sinkhorn-Knopp}
    \end{block}
    }

    
    \small
    \begin{minipage}{0.7\textwidth}
        \begin{center}
        \begin{tikzpicture}
            % words
            \node (a)                   at (0em, 0em)               {een};
            \node (very)                at (5em, 0em)               {heel};
            \node (easy)                at (10em, 0em)              {eenvoudig};
            \node (example)             at (15em, 0em)              {vorbeeld};
            
            % trees
            \visible<2->{
                \node (a_tensor)            at (0em, 2em)               {\textcolor{cyan}{\ps{$\li$}}};}
            \visible<3->{
                \node (a_n)                 at (-1em, 4em)              {\textcolor{red}{\ps{$n_1$}}};}
            \visible<4->{
                \node (a_np)                at (1em, 4em)               {\textcolor{cyan}{\ps{$np_2$}}};
            }
            \visible<5->{
                \node (very_tensor1)        at (5em, 2em)               {\textcolor{cyan}{\ps{$\li$}}};
            }
            \visible<9->{
                \node (very_tensor2)        at (6.5em, 4em)             {\textcolor{cyan}{{\ps{$\li$}}}};
            }
            \visible<6->{
                \node (very_cot)            at (3.5em, 4em)             {\textcolor{red}{{\ps{$\li$}}}};
            }
            \visible<7->{
                \node (very_n1)             at (2.75em, 6em)            {\textcolor{cyan}{\ps{$n_3$}}};
            }
            \visible<8->{
                \node (very_n2)             at (4.25em, 6em)            {\textcolor{red}{\ps{$n_4$}}};
            }
            \visible<9->{
                \node (very_n3)             at (5.75em, 6em)            {\textcolor{red}{\ps{$n_5$}}};
            }
            \visible<9->{
                \node (very_n4)             at (7.25em, 6em)            {\textcolor{cyan}{\ps{$n_6$}}};
            }
            
            \visible<9->{
            \node (easy_tensor)         at (10em, 2em)              {\textcolor{cyan}{\ps{$\li$}}};
            \node (easy_n1)             at (9em, 4em)               {\textcolor{red}{\ps{$n_7$}}};
            \node (easy_n2)             at (11em, 4em)              {\textcolor{cyan}{\ps{$n_8$}}};
            
            \node (ex_n)                at (15em, 2em)              {\textcolor{cyan}{\ps{$n_9$}}};
            }
            
            % internal links -- dynamic
            \visible<2->{
                \draw (a_n)         --      (a_tensor)          --          (a_np);
            }
            \visible<5->{
                \draw (very_cot)    --      (very_tensor1)      --          (very_tensor2);
            }
            \visible<6->{
                \draw (very_n1)     --      (very_cot)          --          (very_n2);
            }
            \visible<9->{
                \draw (very_n3)     --      (very_tensor2)      --          (very_n4);
            }
            \visible<9->{
                \draw (easy_n1)     --      (easy_tensor)       --          (easy_n2);
            }
        \end{tikzpicture}
        \end{center}
    \end{minipage}%
    \begin{minipage}{0.3\textwidth}
        \visible<11->{
        \alt<12->{
        \begin{tikzpicture}[scale=0.4]
            \fill[gray!10] (3, 0) rectangle +(1,1); %
            \fill[gray!80] (3, 1) rectangle +(1,1);
            \fill[gray!5] (3, 2) rectangle +(1,1); %
            \fill[gray!5] (3, 3) rectangle +(1,1); %
            \fill[gray!10] (2, 0) rectangle +(1,1); %
            \fill[gray!5] (2, 1) rectangle +(1,1); %
            \fill[gray!70] (2, 2) rectangle +(1,1);
            \fill[gray!15] (2, 3) rectangle +(1,1); %
            \fill[gray!5] (1, 0) rectangle +(1,1); %
            \fill[gray!5] (1, 1) rectangle +(1,1); %
            \fill[gray!00] (1, 2) rectangle +(1,1); %
            \fill[gray!90] (1, 3) rectangle +(1,1);
            \fill[gray!70] (0, 0) rectangle +(1,1);
            \fill[gray!15] (0, 1) rectangle +(1,1); %
            \fill[gray!5] (0, 2) rectangle +(1,1); %
            \fill[gray!10] (0, 3) rectangle +(1,1); %
            \node[left]     at (0, 0.5) {\textcolor{red}{$n_7$}};
            \node[left]     at (0, 1.5) {\textcolor{red}{$n_5$}};
            \node[left]     at (0, 2.5) {\textcolor{red}{$n_4$}};
            \node[left]     at (0, 3.5) {\textcolor{red}{$n_1$}};
            \node[above]    at (0.5, 4) {\textcolor{cyan}{$n_3$}};
            \node[above]    at (1.5, 4) {\textcolor{cyan}{$n_6$}};
            \node[above]    at (2.5, 4) {\textcolor{cyan}{$n_8$}};
            \node[above]    at (3.5, 4) {\textcolor{cyan}{$n_9$}};
            \draw (0, 0) grid (4,4);
        \end{tikzpicture}}{
        \begin{tikzpicture}[scale=0.4]
            \fill[gray!20] (3, 0) rectangle +(1,1); %
            \fill[gray!60] (3, 1) rectangle +(1,1);
            \fill[gray!10] (3, 2) rectangle +(1,1); %
            \fill[gray!10] (3, 3) rectangle +(1,1); %
            \fill[gray!20] (2, 0) rectangle +(1,1); %
            \fill[gray!10] (2, 1) rectangle +(1,1); %
            \fill[gray!40] (2, 2) rectangle +(1,1);
            \fill[gray!30] (2, 3) rectangle +(1,1); %
            \fill[gray!10] (1, 0) rectangle +(1,1); %
            \fill[gray!10] (1, 1) rectangle +(1,1); %
            \fill[gray!00] (1, 2) rectangle +(1,1); %
            \fill[gray!80] (1, 3) rectangle +(1,1);
            \fill[gray!40] (0, 0) rectangle +(1,1);
            \fill[gray!30] (0, 1) rectangle +(1,1); %
            \fill[gray!10] (0, 2) rectangle +(1,1); %
            \fill[gray!20] (0, 3) rectangle +(1,1); %
            \node[left]     at (0, 0.5) {\textcolor{red}{$n_7$}};
            \node[left]     at (0, 1.5) {\textcolor{red}{$n_5$}};
            \node[left]     at (0, 2.5) {\textcolor{red}{$n_4$}};
            \node[left]     at (0, 3.5) {\textcolor{red}{$n_1$}};
            \node[above]    at (0.5, 4) {\textcolor{cyan}{$n_3$}};
            \node[above]    at (1.5, 4) {\textcolor{cyan}{$n_6$}};
            \node[above]    at (2.5, 4) {\textcolor{cyan}{$n_8$}};
            \node[above]    at (3.5, 4) {\textcolor{cyan}{$n_9$}};
            \draw (0, 0) grid (4,4);
        \end{tikzpicture}}
        }
        \end{minipage}
\end{frame}

%████████████████████████████████
\begin{frame}{Other ingredients: Alpino \& ODWN}

Alpino\footcite{alpino}:
\begin{itemize}
    \item Stochastic Attribute Value Grammar (HPSG) for Dutch
    \item Builds dependency structures
    \item Output converted to ILL$_{\li,\diamondsuit,\Box}$
\end{itemize}
\vfill

Open Dutch WordNet\footcite{ODWN:2016}:%{\smaller[3]\fullcite{ODWN:2016}}
\begin{itemize}
    \item Princeton-style, subset of Cornetto
    \item 52k synsets
    \item Synonymy, hypernymy, antonymy, similarity \& derivation
\end{itemize}


\end{frame}


\section{Natural Tableau}

%████████████████████████████████
\begin{frame}
{Natural Tableau: 
proving\footcite{abzianidze-2015-tableau}
\visible<11->{and \alert{learning}\footcite{abzianidze-2020-learning}}}

\vspace{-6mm}
\begin{minipage}{0.38\pagewidth}
    \smaller
    LangPro:
    \begin{itemize}
        \item Operates on semantic $\lambda_\to$ terms
        \item[] \quad (as semantic \textit{tableaus})
        \item Generic to ent/cont
        \item<2-> Proof by refutation
        \item<3-> Meaning decomposition
        \item<12-> Learning as abduction:\\
        $\synt{hedgehog} \subs \synt{small~animal}$
    \end{itemize}
\end{minipage}%
\begin{minipage}{0.66\pagewidth}
\hspace{-20pt}
\scalebox{.7}{\begin{forest}
baseline,for tree={align=center, parent anchor=south, child anchor=north, l sep=5mm, s sep=18mm}
[{\lab{1}~$\synt{a}~\synt{hedgehog}~(\lambda x.\,\synt{a}~\synt{boy}~(\lambda y.\,\synt{by}~y~\synt{cradle}~x)) : \T$}\\
 {\lab{2}~$\synt{a}~\synt{person}~(\lambda x.\,\alt<11->{\alert{\synt{a}~(\synt{small}~\synt{animal})}}{\synt{an}~\synt{animal}}~(\lambda y.\,\synt{hold}~y~x) : \F$}\toppad{4mm},
 visible on=<2->
 [{\lab{3}~$\alt<12->{\bluetxt{\synt{hedgehog}}}{\synt{hedgehog}} : [h] : \T$}\\
  {\lab{4}~$\synt{a}~\synt{boy}~(\lambda y.\,\synt{by}~y~\synt{cradle}~h) : \T$}\toppad{4mm},
  	labelA={\ndList{1}},
  	%labB={0mm}{\footnotesize}{\ndList{2,$p$}}
  	visible on=<3->
  [{\lab{5}~$\synt{boy} : [b] : \T$}\\
  {\lab{6}~$\synt{by}~b~\synt{cradle} : [h] : \T$}\toppad{4mm},
  labelA={\ndList{4}},
  visible on=<4->
  [{\lab{7}~$\synt{cradle} : [h,b] : \T$},
    labelA={\ndList{6}},
    %labB={0mm}{\footnotesize}{\ndList{2,$b$}},
    node options={label={[label distance=2pt, font=\footnotesize, visible on=<6->]-90:{\ndList{2,$b$}}}},
    visible on=<5->
      [{\lab{8}~$\synt{person} : [b] : \F$},
      visible on=<6->
      [{$\btimes$},
      	  labelA={\ndList{5,8}},
      	  visible on=<7->
      ]
      ]
      [{\lab{9}~$\alt<11->{\alert{\synt{a}~(\synt{small}~\synt{animal})}}{\synt{an}~\synt{animal}}~(\lambda y.\,\synt{hold}~y~b) : \F$},
         %labB={0mm}{\footnotesize}{\ndList{9,$h$}},
         node options={label={[label distance=2pt, font=\footnotesize, visible on=<8->]-90:{\ndList{9,$h$}}}},
         visible on=<6->
      [{\lab{10}~$\alt<11->{\alert{\synt{small}~\synt{animal}}}{\synt{animal}} : [h] : \F$},
        visible on=<8->
         [{\alt<11->{\alert{\textsc{open}}}{$\btimes$}},
          labelA={\ndList{3,10}},
          visible on=<9->
         ]
      ]  
      [{\lab{11}~$\synt{hold} : [h,b] : \F$},
        visible on=<8->
        [{$\btimes$},
          labelA={\ndList{7,11}},
          visible on=<10->
        ]
      ] 
      ]
  ]      
  ]      
 ]       
]
\end{forest}}
\vspace{-10mm}
\end{minipage}

\end{frame}

\fi



%████████
\section{Findings}

%████████████████████████████████
\begin{frame}{Experiments \& Results}



\centerline{\scalebox{.9}{
\begin{tabular}[t]{@{} r@{~/~} l  c  c@{}}
\toprule
Parser&\textsc{pos} tagger 
    & \visible<1->{Test acc.\%} 
    & \visible<2->{($-$\small{abduction})} \\\midrule
Alpino&Alpino
    & \visible<1->{75.9} 
    & \visible<2->{{\smaller[1](-1.8)}} \\
Alpino&spaCy
    & \visible<1->{77.6} 
    & \visible<2->{{\smaller[1](-1.7)}} \\
NPN&Alpino
    & \visible<1->{74.3} 
    & \visible<2->{{\smaller[1](-1.5)}} \\
NPN&spaCy
    & \visible<1->{76.4} 
    & \visible<2->{{\smaller[1](-1.4)}} \\
    \midrule
\multicolumn{2}{l}{\kern-2mm\textbf{\oranjetxt{\ensLP}}} 
    & \visible<1->{\textbf{78.7}} 
    & \visible<2->{{\smaller[1](-1.6)}} \\
\bottomrule
\end{tabular}}
\hspace{5mm}
\raisebox{-10.5mm}{\visible<3->{
    \scalebox{.9}{\begin{tabular}[t]{l|r|r|r|}
    \textbf{\oranjetxt{LP\,2$\times$2}\,\%} & E & C & N \\\hline
    Entail. & \hmc{15} & \hmc{ 0} & \hmc{14} \\\hline
    Contra. & \hmc{ 0}  & \hmc{10} & \hmc{ 5} \\\hline
    Neutral & \hmc{ 1}  & \hmc{ 1} & \hmc{55} \\\hline
\end{tabular}}}}
}
\bigskip

\visible<4->{
tl;dr:
\begin{itemize}
\item Consistent performance across combinations
\item Ensemble outperforms individuals
\item Abduction works well!
% \item NPN slightly inferior to Alpino (..why?)
\end{itemize}
}
    
\end{frame}


%\subsection{Comparison}
%████████████████████████████████
\begin{frame}{Does it beat BERT? \emoji{upside-down-face}
\visible<3->{ \dots sometimes \emoji{smiling-face-with-sunglasses}}}
\vspace{-7mm}

% \begin{center}
% \scalebox{.9}{
% \begin{tabular}[t]{@{}l@{~~~} l l l@{}}%\toprule
% Models
% & All %\visible<2->{\smaller[1]{$\pm\Delta$}} 
% & Ent %\visible<3->{\smaller[1]{$\pm\Delta$}} 
% & Cont %\visible<4->{\smaller[1]{ $\pm\Delta$}}
% \\\midrule
% \ensLP & 78.7 & 50.6 & 66.3
% \\\arrayrulecolor{gray!50}\midrule[.5pt]\arrayrulecolor{black}
% \textbf{BERTje}\footcite{bertje} 
% &   82.0 %\visible<2->{\smaller[1]{$-0.3$}}
% &	86.2 %\visible<3->{\smaller[1]{$+2.0$}}
% &	86.7\\ %\visible<4->{\smaller[1]{$+0.8$}}\\
% mBERT\footcite{bert} 
% &   79.9 %\visible<2->{\smaller[1]{$+0.7$}}
% &	79.0 %\visible<3->{\smaller[1]{$+4.7$}}
% &	81.9\\ %\visible<4->{\smaller[1]{$+3.1$}}\\
% RobBert\footcite{delobelle2020robbert} 
% &   81.7 %\visible<2->{\smaller[1]{$+0.9$}}
% &	76.9 %\visible<3->{\smaller[1]{$+6.4$}}
% &	85.3\\ %\visible<4->{\smaller[1]{$+1.1$}}\\
% \bottomrule
% \end{tabular}}\\
% \end{center}

\begin{center}
\scalebox{.9}{
\begin{tabular}[t]{@{}l@{~~~} c c@{}}%\toprule
Models
& test acc.\%  & \smaller[1]{$\pm\Delta$ LangPro's E\&C}
\\\midrule
\ensLP & 78.7 & $-$
\\\arrayrulecolor{gray!50}\midrule[.5pt]\arrayrulecolor{black}
\textbf{BERTje}\footcite{bertje} 
&   82.0 & \smaller[1]{$-0.3$}\\
mBERT\footcite{bert} 
&   79.9 & \smaller[1]{$+0.7$}\\
RobBert\footcite{delobelle2020robbert} 
&   81.7 & \smaller[1]{$+0.9$}\\
\bottomrule
\end{tabular}}
%
%
\visible<2->{
\raisebox{-1mm}{\scalebox{.57}{
\begin{tabular}[t]{r r}
    \begin{tabular}[t]{r|r|r|r|}
        \textbf{\oranjetxt{LP\,2$\times$2}\,\%} & E & C & N \\\hline
        E & \hmc{15}  & \hmc{ 0} & \hmc{14} \\\cline{2-4}
        C & \hmc{ 0}  & \hmc{10} & \hmc{ 5} \\\cline{2-4}
        N & \hmc{ 1}  & \hmc{ 1} & \hmc{55} \\\cline{2-4}
    \end{tabular}
&  
    \begin{tabular}[t]{r|r|r|r|}
        \textbf{BERTje\,\%} & E & C & N \\\hline
        E & \hmc{25}  & \hmc{ 0} & \hmc{ 4} \\\cline{2-4}
        C & \hmc{ 1}  & \hmc{13} & \hmc{ 1} \\\cline{2-4}
        N & \hmc{10}  & \hmc{ 3} & \hmc{45} \\\cline{2-4}
    \end{tabular}\bigskip
\\
    \begin{tabular}[t]{r|r|r|r|}
        \textbf{mBERT\,\%} & E & C & N \\\hline
        E & \hmc{23}  & \hmc{ 0} & \hmc{ 6} \\\cline{2-4}
        C & \hmc{ 1}  & \hmc{12} & \hmc{ 2} \\\cline{2-4}
        N & \hmc{ 9}  & \hmc{ 2} & \hmc{45} \\\cline{2-4}
    \end{tabular}
&  
    \begin{tabular}[t]{r|r|r|r|}
        \textbf{RobBert\,\%} & E & C & N \\\hline
        E & \hmc{22}  & \hmc{ 0} & \hmc{ 7} \\\cline{2-4}
        C & \hmc{ 1}  & \hmc{12} & \hmc{ 2} \\\cline{2-4}
        N & \hmc{ 7}  & \hmc{ 2} & \hmc{47} \\\cline{2-4}
    \end{tabular}
\\
\end{tabular}
}}}
%
\end{center}

\visible<3->{
\begin{tabular}{l@{\hspace{7mm}} l}
\textsmaller[1]{
\nlip{1556}{Entailment}
{Een man draagt een \alert{boom}}
{Een man draagt een \alert{plant}}}
&
\textsmaller[1]{
\nlip{175}{Entailment}
{Een familie kijkt naar een kleine jongen die een honkbal \alert{raakt}}
{Een jongen \alert{slaat} een honkbal}}
\\[7mm]
\textsmaller[1]{
\nlip{4092}{Contradiction}
{De persoon tekent \alert{niet}}
{Een man tekent een foto}}
&
\textsmaller[1]{
\nlip{6356}{Contradiction}
{Een vrouw in een rode jurk is een instrument aan het \alert{opbergen}}
{Een vrouw in een rode jurk \alert{bespeelt} een instrument}}
\end{tabular}
}

% All 13 E&C problems which were solved only by Langpr0
% 168	A child is hitting a baseball	A child is missing a baseball	3.8	CONTRADICTION
% 175	A family is watching a little boy who is hitting a baseball	A boy is hitting a baseball	4.2	ENTAILMENT
% 177	A family is watching a little boy who is hitting a baseball	A child is hitting a baseball	4.1	ENTAILMENT
% 798	Two people are in the snow not wearing clothes, which provides camouflage. 	Two people are in the snow, wearing clothes that provide camouflage	3.8	CONTRADICTION
% 1158	A woman is removing ingredients from a bowl	A woman is adding ingredients to a bowl	3.3	CONTRADICTION
% 1556	A man is carrying a tree	A man is carrying a plant	4.3	ENTAILMENT
% 3606	A monkey is riding a bike	A monkey is riding a bicycle	4.9	ENTAILMENT
% 4092	The person is not drawing	A man is drawing a picture	3.5	CONTRADICTION
% 4470	A man is kicking a soccer ball	A man is kicking a ball	4.8	ENTAILMENT
% 4925	Someone is putting ingredients into a wok	Someone is putting ingredients into a pan	4.5	ENTAILMENT
% 5140	A man is having fun with water	A man is playing with water	4.5	ENTAILMENT
% 6356	A woman in a red dress is putting away an instrument	A woman in a red dress is playing an instrument	3.7	CONTRADICTION
% 9480	A black dog and a white dog are wrestling on the ground	Two dogs are fighting	4.2	ENTAILMENT


\end{frame}


%████████████████████████████████
\begin{frame}{SICK NL}

More challenging than the original:
\begin{itemize}
    \item[1] Phrase shifts from MT\\
    \begin{tabular}{ccc}
         \emph{drawing a picture} & $\mapsto$ &  
         \oranjetxt{\emph{een tekening maakt}} \textbar{} \oranjetxt{\emph{tekent een foto}}\\
         \emph{dirt bike race} & $\mapsto$ & \oranjetxt{\emph{crossmotorwedstrijd}} \textbar{} \oranjetxt{\emph{crossmotorrace}}
    \end{tabular} 
    \item[2]<2-> Good translations, bad labels? \\
    \begin{tabular}{ccc}
    \textsmaller[1]{\nlip{3181}{Neutral?}
    {A man is trekking in the woods}
    {The man is not hiking in the woods}} 
    & $\mapsto$ &
    \textsmaller[1]{\nlip{3181}{\alert{Neutral}}
    {\oranjetxt{Een man is aan het wandelen in het bos}}
    {\oranjetxt{De man is niet aan het wandelen in het bos}}}
    \end{tabular}
\end{itemize}
\vfill

\visible<3->{
..but contains useful lexical relations:
    \begin{itemize}\itemsep0mm
    \item[] \synt{\oranjetxt{pizza}}\subs\synt{\oranjetxt{voedsel}}/\!\transl{food}
    \item[] \synt{\oranjetxt{lopen}} $\equiv$ \synt{\oranjetxt{rennen}}
    \item[] $\ldots$
    \end{itemize}
}
\end{frame}

\section{Conclusion}
%████████████████████████████████
\begin{frame}{Conclusion \& Future work}

\begin{itemize}
    \item[\emoji{check-mark}] Lots of $\lambda$ambdas
    \item[\emoji{check-mark}] First multilingual application of LangPro
    \item[\emoji{check-mark}] First ``real-world'' use of NPN
    \item[\emoji{check-mark}] Explainable reasoning for Dutch
\end{itemize}   

\vfill

\visible<2->{
\begin{itemize}
    \item[\emoji{memo}] Qualitative comparison of (more) parsers
    \item[\emoji{memo}] Experiment with more languages
    \item[\emoji{memo}] Neural tableaus? 
\end{itemize}
}
\end{frame}

%████████████████████████████████
\appendix
\section{Backup sldies}
% BACKUP SLIDES
%████████████████████████████████
\begin{frame}{Syntactic $\lambda$-terms to $\lambda$-logical forms}

$\synt{play}^{\diamondsuit^{\text{su}}\np\multimap\smain}
\left(
    \depDI{\text{su}}
    \left(
        \left(
            \depBE{\text{det}}~\synt{a}^{\Box^\text{det}(\n \multimap \np)}
        \right)
        \synt{boy}^\n
    \right)
\right)$
\hfill $\leadsto ~ 
\synt{play}^{\np,\s}
    \left(
        \synt{a}^{\n,\np} ~ \synt{boy}^\n
    \right)
$\vfill

$\synt{large}^{\np,\np}~\big(\synt{brown}^{\np,\np}~(\synt{a}^{\n,\np}~\synt{dog}^{\n})\big)$
\hfill $\leadsto ~ 
\synt{a}^{\n,\np}~\big(\synt{large}^{\hlt{\n,\n}}~(\synt{brown}^{\hlt{\n,\n}}~\synt{dog}^{\n})\big)
$\vfill

$\synt{and}~\big(\abst{x} \synt{brown}  (x~\synt{dog})\big) \big(\abst{y} \synt{black}  (y~\synt{dog})\big)~\synt{no}$\\\medskip
\hfill $\leadsto ~ 
\synt{and}^{\hlt{\np,\np,\np}}~\big(\hlt{\synt{no}}~(\hlt{\synt{brown}}~\synt{dog})\big)~\big(\hlt{\synt{no}}~(\hlt{\synt{black}}~\synt{dog})\big)
$\vfill

$
\synt{cut}^{\pp,\n,\np,\s}~(\synt{in}^{\n,\pp}~\synt{slice}_\n)~\synt{meat}_\n$
\hfill $\leadsto ~ 
\synt{cut}^{\pp,\hlt{\np},\np,\s}~(\synt{in}^{\hlt{\np},\pp}~\hlt{\{}\synt{slice}^\n\hlt{\}^\np})~\hlt{\{}\synt{meat}^{\n}\hlt{\}^\np}
\label{r:n_to_np}
$
\bigskip

\begin{itemize}
\item Map POS tags and shift to slightly \emph{Generalized} POS tags: UPOS \& Penn
\item Use only these syntactic categories: $\n, \np_x, \s_x, \pp, \pr$
\item Function words $\mapsto$ canonical terms (excl. prepositions)\kern-30mm
\end{itemize}
    
\end{frame}


%████████████████████████████████
\begin{frame}{\hspace{85mm}Natural Tableau for Dutch \emoji{flag-netherlands}}

\begin{minipage}[t]{80mm}
\vspace{-15.3mm}
\scalebox{.62}{\begin{forest}
baseline,for tree={align=center, parent anchor=south, child anchor=north, inner sep=0pt, l sep=3mm, s sep=30mm}
[{$
          \oranjetxt{\sysm{vrolijk}}
          \left(\oranjetxt{\sysm{een}}~\oranjetxt{\sysm{jongen}}
            \left(\abst{x}\oranjetxt{\sysm{een}}~\oranjetxt{\sysm{piano}}
              \left(\abst{y} \oranjetxt{\sysm{spelen}} ~ y ~ x \right)\right)\right): \T$}\\
         {\lab{1}~$\sysm{happily}
          \left(\sysm{a}~\sysm{boy}
            \left(\abst{x}\sysm{a}~\sysm{piano}
              \left(\abst{y} \sysm{play} ~ y ~ x \right)\right)\right): \T$}\\
 {$
    \oranjetxt{\sysm{een}}~\oranjetxt{\sysm{piano}} 
      \left(\oranjetxt{\sysm{worden}}
        \left(\abst{y}\oranjetxt{\sysm{een}}~\oranjetxt{\sysm{persoon}}
          \left(\abst{x}\oranjetxt{\sysm{door}} ~ x ~ \oranjetxt{\sysm{bespelen}} ~ y \right)\right)\right) : \F$}\toppad{4mm}\\
    {\lab{2}~$\sysm{a}~\sysm{piano} 
      \left(\sysm{be}
        \left(\abst{y}\sysm{a}~\sysm{person}
          \left(\abst{x}\sysm{by} ~ x ~ \sysm{play} ~ y \right)\right)\right) : \F$}
 [{\lab{3}~$
          \sysm{een}~\sysm{jongen}
            \left(\abst{x}\sysm{een}~\sysm{piano}
              \left(\abst{y} \sysm{spelen} ~ y ~ x \right)\right) : \T$},
  labelA={\ndList{1}},
 [{\lab{4}~$\sysm{jongen} : [j] : \T$}\\
  {\lab{5}~$
    \sysm{een}~\sysm{piano}
      \left(\abst{y} \sysm{spelen} ~ y ~ j \right) : \T$}\toppad{4mm},
  	labelA={\ndList{3}},
  	%labB={0mm}{\footnotesize}{\ndList{2,$p$}}
  [{\lab{6}~$\sysm{piano} : [p] : \T$}\\
   {\lab{7}~$
      \sysm{spelen} ~: [p, j] : \T$}\toppad{4mm},
   labelA={\ndList{5}},
   [{\lab{8}~$
        \sysm{worden}
        \left(\abst{y}\sysm{een}~\sysm{persoon}
          \left(\abst{x}\sysm{door} ~ x ~ \sysm{bespelen} ~ y \right)\right) : [p] :\F$},
    labelA={\ndList{2,6}},
    [{\lab{9}~$
        \abst{y}\sysm{een}~\sysm{persoon}
          \left(\abst{x}\sysm{door} ~ x ~ \sysm{bespelen} ~ y \right) : [p] :\F$},
     labelA={\ndList{8}},
     [{\lab{10}~$
        \sysm{een}~\sysm{persoon}
          \left(\abst{x}\sysm{door} ~ x ~ \sysm{bespelen} ~ p \right) :\F$},
      labelA={\ndList{9}}, labB={0mm}{\footnotesize}{\ndList{10,$j$}}
      [{\lab{11}~$
        \sysm{persoon} : [j] : \F$},
       [{$\btimes$},
           labelA={\ndList{4,11}} 
       ]
      ]
      [{\lab{12}~$
        \abst{x}\sysm{door} ~ x ~ \sysm{bespelen} ~ p : [j] :\F$},
       [{\lab{13}~$
        \sysm{door} ~ j ~ \sysm{bespelen} ~ p : \F$}, labelA={\ndList{12}}
        [{\lab{14}~$
         \sysm{bespelen} [p, j] : \F$}, labelA={\ndList{13}}
         [{$\btimes$}, labelA={\ndList{7,14}} 
         ]
        ]
       ]
      ]
     ]
    ]
   ]      
  ]      
 ] 
 ]
]
\end{forest}}
\end{minipage}
%
%
%
\begin{minipage}[t]{50mm}
\vspace{15mm}
\textsmaller[1]{
\nlip{1771}{Entailment}
{\oranjetxt{Een jongen speelt vrolijk piano}}
{\oranjetxt{Een piano wordt bespeeld door een persoon}}
}

\textsmaller[1]{
\nlip{1771}{Entailment}
{A boy is happily playing the piano}
{A piano is being played by a person}
}
\end{minipage}

\end{frame}

%████████


\end{document}
