\documentclass{beamer}	
\usecolortheme{wolverine}

\setbeamertemplate{caption}{\raggedright\insertcaption\par}

\usepackage{tabularx}
\usepackage{booktabs}
\usepackage{multirow}
\newcolumntype{C}{>{\centering\arraybackslash}X}
\usepackage{proof}
\usepackage{amsmath}
\usepackage{wasysym}
\usepackage{tikz}
\usepackage{qtree}
\usetikzlibrary{arrows.meta,positioning,matrix}

\newcommand{\term}[1]{\texttt{#1}}
\newcommand{\li}{\!\multimap\!}
\newcommand{\lotimes}{\!\otimes\!}
\newcommand{\loplus}{\!\oplus\!}
\newcommand{\lin}[1]{\langle#1\rangle}
\newcommand{\lint}[1]{[#1]}


\title{The Grammar of Grammars}
\author{K. Kogkalidis}
\institute{Logic \& Language 2020}


\begin{document}
\date{}
\maketitle

\begin{frame}{Recap: Formal Grammars}
	\small
	\begin{block}{Formal Grammars}
	A formal grammar $\mathcal{G}$ is a tuple $\mathcal{G} = \langle V, \Sigma, R, S \rangle$, where
	\begin{itemize}
		\item[$V$] the \textit{vocabulary}, a set of symbols
		\item [$\Sigma$] the set of \textit{terminal} symbols, $\Sigma \subset V$
		\item [$R$] the set of \textit{production rules}, $R \subset V^*\!\times\! V^*$\\
		\item [$S$] the \textit{initial symbol}, $S \in V - \Sigma$ 
	\end{itemize}
	\end{block}
	
	\begin{block}{Rules}
		A rule $r \in R$ is usually written as $\alpha \to \beta$, where $\alpha$, $\beta$ \textit{strings}	of $V$,
		i.e. $\alpha, \beta \in V^*$.
		
		Allowing only specific forms of rules $R$ leads to a \alert{hierarchy} of formal grammars, each with their own expressivity and complexity.
	\end{block}
	
	\begin{block}{Language}
		The set of \textit{words} (strings) $\mathcal{L}_\mathcal{G} \in \Sigma^*$ that can be generated by $\mathcal{G}$.
	\end{block}
\end{frame}

\begin{frame}{Chomsky Hierarchy}
	\footnotesize
    \begin{tabularx}{0.99\textwidth}{@{}llll@{}}
    type & grammar & automaton & rule form\\
    \toprule
    3 & regular & finite state machine & $A\to a; A\to a B$\\
    2 & context-free & pushdown automaton & $A\to \gamma$\\
    1 & context-sensitive & linear bounded automaton & $\alpha A \beta \to \alpha \gamma \beta$ \\
    0 & recursively enumerable & Turing machine & $\alpha \to \beta$
	\end{tabularx}
	
	\begin{flushright}
		$A,B$: non-terminals, $a$: terminal, $\alpha, \beta, \gamma$: strings of $V$
	\end{flushright}
	
	\alert{
	\[
		\text{Type-}3 \subset \text{Type-}2 \subset \text{Type-}1 \subset \text{Type-}0
	\]
	}
\end{frame}

\begin{frame}{Natural Language}
	\begin{itemize}
		\item $R$ aligned with speech, phonology, morphology
		\item $CF$ captures most syntactic patterns (but not all!)
		\item $CS$ too expressive and complex to be of real use
		\item [$\rightsquigarrow$] need a better charting between $CF$ and $CS$
	\end{itemize}
\end{frame}

\begin{frame}{Pumping Lemma for CFL}
	\small 
	Let $\mathcal{G}=\langle V, \Sigma, R, S \rangle$ a CFG generating an infinite language $\mathcal{L}_\mathcal{G}$.	
	
	\begin{align*}
		&\exists \ k \in \mathbb{N}: \\
		& \quad \forall \ w \in \mathcal{L}_\mathcal{G} \land |w| \geq k:\\
		& \quad \quad \exists \ x,y,z,v_1,v_2 \in \Sigma^*:\\
		& \quad \quad \quad \bigwedge \Big\{  w = xv_1yv_2z, \ |v_1v_2|\geq 1, \ |v_1yv_2| \leq k,\\
		& \quad \quad \quad \quad \quad \ \forall \ i \in \mathbb{N}: \{ xv_1^iyv_2^iz \in \mathcal{L}_\mathcal{G} \}\Big\}
	\end{align*}
	\pause
	
	\begin{block}{Example}
	The copy language $\mathcal{L} = \{ww \ | \ w \in \{a,b\}^*\}$ is \alert{not} context-free, but similar constructions occur
	in natural language (crossing dependencies):
	\begin{center}
		\dots \textit{dat \textcolor{blue}{Wim} \textcolor{green}{Jan} \textcolor{red}{Marie} \textcolor{purple}{de kinderen}
		\textcolor{blue}{zag} \textcolor{green}{helpen} \textcolor{red}{leren} \textcolor{purple}{zwemmen}} \dots\\
		``\dots that \textcolor{blue}{Wim saw} \textcolor{green}{Jan help} \textcolor{red}{Marie teach} \textcolor{purple}{the kids how to swim} \dots''
	\end{center}
	\end{block}
\end{frame}

\begin{frame}{The landscape beyond CFL}
	\small 

	\begin{minipage}{0.65\textwidth}
	The class of \alert{mildly context-sensitive languages}:
	\begin{itemize}
		\item contains context-free languages
		\item capture a finite number of cross-serial dependencies, i.e languages of the form: $\mathcal{L} = \{w^k | \ w \in \Sigma^*\}$ for some $k$
		\item maintains polynomial parsing time (CFGs have $\mathcal{O}(n^3)$)
		\item is characterized by constant growth: word length increase is linear-bound
	\end{itemize}	
	\end{minipage}%
	\begin{minipage}{0.35\textwidth}
	\begin{figure}
	\includegraphics[width=1\textwidth]{mcs.pdf}	
	\end{figure}
	\end{minipage}
	\vfill
	
\end{frame}

\begin{frame}{Abstract Categorial Grammars}
	\small
	\alert{Abstract Categorial Grammars} model the landscape of formal grammars as a morphism between two ILL${}_\multimap$ logics:
	\[
		\begin{array}{@{}c@{\quad}c@{\quad}c@{}}
			\text{ILL}_\multimap^A & \xrightarrow{\makebox[1in]{$h$}} & \text{IL}_\multimap^{A^\prime}\\
			\text{Source} & \textit{Homomorphism} & \text{Target}
		\end{array}
	\]
	\begin{itemize}
		\item source\\
		logic describing the abstract function-argument structure of the language (tectogrammar)
		\item target\\
		logic describing the concrete surface materialization of the language: strings, trees, etc (phenogrammar)
	\end{itemize}
\end{frame}

\begin{frame}{Abstract Categorial Grammars}
	\small 
	
	\alt<2>{
		\begin{block}{ACG}
			An \alert{abstract categorial grammar} is a tuple $\langle \Sigma_1, \Sigma_2, \mathfrak{L}, s \rangle$, where:
			\begin{itemize}
				\item[$\Sigma_1$] the abstract vocabulary
				\item[$\Sigma_2$] the object language
				\item[$\mathfrak{L}$] the map $\Sigma_1 \to \Sigma_2$
				\item[$s$] the initial or distinguished type, $s \in \mathcal{T}_{\mathcal{A}_1}$
			\end{itemize}
		\end{block}
		
		From the vocabularies we obtain \alert{languages} $\mathcal{L}_1, \mathcal{L}_2$:
		\begin{itemize}
			\item[$\mathcal{L}_1$] the abstract language 
			\[
				\mathcal{L}_1 = \{ t \in \Lambda_{\Sigma_1}  | \ t \text{ an inhabitant of }s\}
			\]
			\item[$\mathcal{L}_2$] the object language 
			\[
				\mathcal{L}_2 = \{ t \in \Lambda_{\Sigma_2} \ | \ \exists \  u \in \mathcal{L}_1: t \text{ the $\hat{\theta}$-image of $u$}\}
			\]
		\end{itemize}
	}{
	\small
	\begin{block}{Vocabulary}
		A vocabulary $\Sigma$ is a ``higher-order linear signature'' $\Sigma = \langle \mathcal{A}, C, \tau \rangle$, where:
		\begin{itemize}
			\item[$\mathcal{A}$] a set of atomic types ($\mathcal{T}_\mathcal{A}$ the type universe)
			\item[$C$] a set of constants ($\Lambda_\Sigma$ the set of well-formed $\lambda$-terms)
			\item[$\tau$] a mapping $C\to \mathcal{T}_\mathcal{A}$
			\end{itemize}
	\end{block}
	
	\begin{block}{Lexicon}
		A lexicon $\mathfrak{L}$ is a mapping $\Sigma_1 \to \Sigma_2$ consisting of $\langle \eta, \theta \rangle$, where
		\begin{itemize}
			\item[$\eta$] a mapping $\mathcal{A}_1 \to \mathcal{T}_{\mathcal{A}_2}$, deriving the homomorphic extension $\hat{\eta}: \mathcal{T}_{\mathcal{A}_1} \to \mathcal{T}_{\mathcal{A}_2}$
%			 $\hat{\eta}(\alpha \li \beta) = \eta(\alpha) \li \eta(\beta)$,
			\item[$\theta$] a mapping $C_1 \to \Lambda_{\Sigma_2}$, deriving the homomorphic extension $\hat{\theta}: \Lambda_{\Sigma_1} \to \Lambda_{\Sigma_2}$
		\end{itemize}
		such that $\vdash \theta(c): \hat{\eta}(\tau(c))$, i.e. $\theta$ respects typing
	\end{block}
	}
	
\end{frame}

\begin{frame}{Example: ACG for the Dyck Language}
	\small
	\begin{block}{Dyck Language}
		The language of well-bracketed parentheses, captured by the CFG:
		\[
			S \to SS {}_{(\mathrm{R}_1)} \ | \  [S]  {}_{(\mathrm{R}_2)} \ | \ \epsilon {}_{(\mathrm{R}_3)}
		\]
	\end{block}

	\alert{Source Signature} $\Sigma_1 = \langle \mathcal{A}_1, C_1, \tau_1 \rangle$
	{\footnotesize
	\[
	\mathcal{A}_1 = \{S\}  
	\quad  C_1= \{\mathrm{R}_1,\ \mathrm{R}_2,\ \mathrm{R}_3\} 
	\quad \tau_1 = \{ \mathrm{R}_1 \mapsto S\li S\li S, 
		\ \mathrm{R}_2 \mapsto S\li S,
		\ \mathrm{R}_3 \mapsto S\}
	\]}

	\alert{Target Signature} $\Sigma_2 = \langle \mathcal{A}_2, C_2, \tau_2 \rangle$
	{\footnotesize
	\[
	\mathcal{A}_2 = \{ * \}
	\quad  C_2= \{ \texttt{[},\ \texttt{]} \} 
	\quad \tau_2 = \{ \texttt{[} \ \mapsto * \li *,\ \texttt{]} \ \mapsto * \li *\}
	\]
	\begin{flushright}
		where $*$ a primitive type s.t. $\texttt{str}=*\li *$
		$\cdot: \texttt{str}\li \texttt{str}\li \texttt{str} = \lambda f.\lambda g.\lambda i.f(g \ i)$
	\end{flushright}		
	}
	
	\alert{Translation} $\mathfrak{L} = \langle \eta, \theta \rangle$
	{\footnotesize
	\[
	\eta = \{ S \mapsto \texttt{str}\} \quad 
	\theta = \{ \mathrm{R}_1 \mapsto \lambda x \lambda y.x\cdot y,
	 \ \mathrm{R}_2 \mapsto \lambda x.\texttt{[} \cdot x \cdot \texttt{]},
	 \ \mathrm{R}_3 \mapsto \lambda x.x
	\}
	\]
	}
\end{frame}

\begin{frame}{Example: ACG for the Dyck Language}
	\small
	\alert{Parsing}\\
		\[
		\texttt{[][[]]} \in ? \mathcal{L}_2
		\quad 		
		\Leftrightarrow
		\quad
		\exists ? u \in \mathcal{L}_1 . \hat{\theta}(u) = \texttt{[][[]]}
		\]
	
	\begin{minipage}[t]{0.1\textwidth}
	\tiny
	\Tree
	[.S
		[.S
			[.S
				$\epsilon$
			]
		]
		[.S
			[.S
				[.S
					$\epsilon$
				]
			]
		]
	]		
	\end{minipage}%
	\quad\quad
	\begin{minipage}[t]{0.8\textwidth}
	\tiny
	\[
		\infer[\li E]{ 
			\vdash ( \mathrm{R}_1  ( \mathrm{R}_2 \mathrm{R}_3) ) ( \mathrm{R}_2  ( \mathrm{R}_2  \mathrm{R}_3 )): S}{
			\infer[\li E]{ \mathrm{R}_1  (\mathrm{R}_2 \mathrm{R}_3): S\li S}{
				{\mathrm{R}_1: S\li S\li S}
				&
				\infer[\li E]{ \mathrm{R}_2 \mathrm{R}_3: S}{
					\mathrm{R}_2: S\li S
					&
					\mathrm{R}_3: S			
				}
			}
			&
			\infer[\li E]{\mathrm{R}_2 (\mathrm{R}_2 \mathrm{R}_3): S}{
				{\mathrm{R}_2: S\li S}
				&
				\infer[\li E]{ \mathrm{R}_2 \mathrm{R}_3: S}{
					{\mathrm{R}_2: S\li S}
					&
					{ \mathrm{R}_3: S}
				}
			}
		}
	\]	
	\end{minipage}
	\vfill

	\[
	\begin{aligned}
u&=( \mathrm{R}_1   ( \mathrm{R}_2 \mathrm{R}_3) ) \ ( \mathrm{R}_2  ( \mathrm{R}_2  \mathrm{R}_3 ))\\
\hat{\theta}(u) &= 
			( \theta(\mathrm{R}_1)  \ ( \theta(\mathrm{R}_2)\ \theta(\mathrm{R}_3)) ) \ ( \theta(\mathrm{R}_2) \  ( \theta(\mathrm{R}_2) \  \theta(\mathrm{R}_3) ))\\
&= \dots \\
&\stackrel{\beta}{\rightsquigarrow} \texttt{[][[]]}
	\end{aligned}
	\]
\end{frame}

\begin{frame}{ACG Hierarchy}
	\small
	{\footnotesize
	The \alert{order} $\mathcal{O}$ of a type $T$ is 
	$\mathcal{O}(T) = 
			\begin{cases}
				0 & T \in \mathcal{A} \\ 
				\mathrm{max}\left( {\mathcal{O}(A) + 1, \mathcal{O}(B)} \right)
				& T = A\li B
	\end{cases}$}
	\vfill	
	
	\pause
	ACG \alert{measures of complexity}:
	\begin{itemize}
		\item Complexity of abstract signature:
			$\mathcal{C}(\Sigma_1) = \mathrm{max}_{c\in {C_1}}\{\mathcal{O}\left(\tau\left(c\right)\right)\}$
		\item Complexity of interpretation:
			$\mathcal{C}(\mathfrak{L}) = \mathrm{max}_{\alpha \in \mathcal{A}_1}\{ \mathcal{O}\left(\eta\left(\alpha\right)\right)\}$
	\end{itemize}
	\begin{flushright}
	The \alert{type} of an ACG is the tuple $\left(\mathcal{C}(\Sigma_1), \mathcal{C}(\mathfrak{L}) \right)$.
	\end{flushright}
	\vfill

	\pause
	\alert{Embedding the Chomsky Hierarchy} \\
	\vfill	
	\centering
	\begin{tabularx}{0.7\textwidth}{cc}
		ACG Type & $\mathcal{L}_2$ Class\\
		\midrule
		(2, 1) & regular\\
		(2, 2) & context-free\\
		(2, 3) & well-nested mildly context-sensitive\\
		(2, n $\geq$ 4) & mildly context-sensitive
	\end{tabularx}
	\vfill
	
\end{frame}

\begin{frame}{Example: m-CFGs in ACG}
	\small
	Multiple context-free grammars operate on \alert{tuples} of strings; tuples can be encoded as higher-order $\lambda$-terms:	
	\[
		\langle a_1, \dots, a_n \rangle \rightsquigarrow \lambda t.(t \ a_1 \dots a_n): \texttt{str}^{(n)}\equiv (\underbrace{\texttt{str} \li \dots \li \texttt{str}}_{n+1}) \li \texttt{str}
	\]

	The language $\{a^nb^nc^nd^n \ | \ n > 0 \}$ is generated by the 2-CFG:
	\[
		S(xy) \to A(x, y) {}_{(\mathrm{R}_1)} \quad A(\term{a}x\term{b}, \term{c}y\term{d}) \to A(x,y) {}_{(\mathrm{R}_2)} \quad A(\epsilon, \epsilon) \to \epsilon {}_{(\mathrm{R}_3)}
	\]	
	
	\alert{ACG encoding}
	\begin{align*}
	& \Sigma_1 = \{A, S\} \quad\quad \ \tau_1 = \{\mathrm{R}_1 \mapsto A\li S,\ \mathrm{R}_2 \mapsto A\li A,\ \mathrm{R}_3 \mapsto A \} \\
	& \Sigma_2 = \{*\}, \quad\quad\quad \tau_2 = \{\texttt{a}, \texttt{b}, \texttt{c}, \texttt{d} \mapsto \texttt{str} \}\\
	& \eta = \{S \mapsto \texttt{str}, A \mapsto \texttt{str}^{(2)} \} \\
	& \theta = \{\mathrm{R}_1 \mapsto \lambda \rho .\left( \rho\ \lambda xy.\left(x+y\right) \right): \texttt{str}^{(2)} \li \texttt{str}, \\
	& \quad\quad \ \mathrm{R}_2 \mapsto \lambda \rho q. \left( \rho \ \lambda xy.\left( q \ \left( a+x+b \right) \ \left( c+y+d \right) \right) \right): \texttt{str}^{(2)}\li \texttt{str}^{(2)},\\
	& \quad\quad \ \mathrm{R}_3 \mapsto \lambda t.(t\  \epsilon\ \epsilon): \texttt{str}^{(2)}
	\}
	\end{align*}
\end{frame}

\begin{frame}{Intermezzo}
	Convincing ourselves about $\theta$
	\begin{align*} 
		\left( ``logic", ``language" \right) &\rightsquigarrow \lambda t.(t \ ``logic" \ ``language") \\
		\theta(\mathrm{R}_1) & = \lambda \rho .\left( \rho\ \lambda xy.\left(x+y\right) \right)
	\end{align*}
	\pause	
	
	\footnotesize
	\begin{align*}
	\theta(\mathrm{R}_1) \ \left( ``logic", ``language" \right) 
	&= \lambda \rho .\left( \rho\ \lambda xy.\left(x+y\right) \right) \ \lambda t.(t \ ``logic" \ ``language") \\ 
	\visible<3->{
	&\stackrel{\beta}{\rightsquigarrow}	\lambda t.(t \ ``logic" \ ``language") \ \lambda xy.(x+y) \\
	\visible<4->{
	&\stackrel{\beta}{\rightsquigarrow} \lambda xy.(x+y) \ ``logic" \ ``language" \\
	\visible<5->{
	&\stackrel{\beta}{\rightsquigarrow} \lambda y.(``logic" + y) \ ``language" \\
	\visible<6->{
	&\stackrel{\beta}{\rightsquigarrow} ``logic" + ``language"
	}}}}
	\end{align*}
\end{frame}

\begin{frame}{Example: m-CFGs in ACG (cont)}
	\small
	\alert{Parsing}\\
	\[
	\texttt{aabbccdd} \in ? \mathcal{L}_2
	\quad 		
	\Leftrightarrow
	\quad
	\exists ? u \in \mathcal{L}_1 . \hat{\theta}(u) = \texttt{aabbccdd}
	\]
	
	\tiny
	\[
		\infer[\li E]{\vdash \mathrm{R}_1\left(\mathrm{R}_2 \left(\mathrm{R}_2 \mathrm{R}_3 \right) \right): S}{
			{\mathrm{R}_1: A\li S}
			&
			\infer[\li E]{\mathrm{R}_2\left(\mathrm{R}_2 \mathrm{R}_3 \right): A}{
				{\mathrm{R}_2: A\li A}
				&
				\infer[\li E]{\mathrm{R}_2 \mathrm{R}_3: A}{
					{\mathrm{R}_2: A\li A}
					&
					{\mathrm{R}_3: A}
				}
			}
		}
	\]	
	\vfill
	
	
	\footnotesize
	\begin{align*}
		\theta(\mathrm{R}_2) \theta(\mathrm{R}_3) &= \lambda pq.(p \ \lambda xy.(q \ (\texttt{a}+x+\texttt{b}) \ (\texttt{c}+y+\texttt{d}))) \ \lambda t.(t \ \epsilon \ \epsilon)\\
	&\stackrel{\beta}{\rightsquigarrow} \lambda q.(\lambda t.(t \ \epsilon \ \epsilon) \ \lambda xy.(q \ (\texttt{a}+x+\texttt{b}) \ (\texttt{c}+y+\texttt{d}))) \\
	&\stackrel{\beta}{\rightsquigarrow} \lambda q.(\lambda xy.(q \ (\texttt{a}+x+\texttt{b}) \ (\texttt{c}+y+\texttt{d})) \ \epsilon \ \epsilon)\\
	&\stackrel{\beta}{\rightsquigarrow} \lambda q.(q \ (\texttt{a}+\epsilon+\texttt{b}) \ (\texttt{c}+\epsilon+\texttt{d})) \stackrel{\beta}{\rightsquigarrow} \lambda q.(q \ \texttt{ab} \ \texttt{cd}) \\
		\visible<2->{
		\theta(\mathrm{R}_2) (\theta(\mathrm{R}_2) \theta(\mathrm{R}_3)) &= \lambda fg.(f \ \lambda xy.(g \ \texttt{a}+x+\texttt{b}) \ (\texttt{c}+y+\texttt{d}))) \ \lambda q.(q \ \texttt{ab} \ \texttt{cd})\\
	&\stackrel{\beta}{\rightsquigarrow} \lambda g.(\lambda q.(q\  \texttt{ab} \ \texttt{cd}) \ \lambda xy.(g \ (\texttt{a}+x+\texttt{b}) \ (\texttt{c}+y+\texttt{d})))\\
	&\stackrel{\beta}{\rightsquigarrow} \lambda g.(\lambda xy.(g \ (\texttt{a}+x+\texttt{b}) \ (\texttt{c}+y+\texttt{d})) \ \texttt{ab} \ \texttt{cd}) \\
	&\stackrel{\beta}{\rightsquigarrow} \lambda g.(g \ (\texttt{a}+\texttt{ab}+\texttt{b}) \ (\texttt{c}+\texttt{cd}+\texttt{d})) 	\stackrel{\beta}{\rightsquigarrow} \lambda g.(g \ \texttt{aabb} \ \texttt{ccdd})
		}
	\end{align*}
	\vfill 
	
\end{frame}

\end{document}