\documentclass{beamer}	
\usecolortheme{wolverine}

\setbeamertemplate{caption}{\raggedright\insertcaption\par}

\usepackage{tabularx}
\usepackage{booktabs}
\usepackage{multirow}
\newcolumntype{C}{>{\centering\arraybackslash}X}
\usepackage{proof}
\usepackage{amsmath}
\usepackage{wasysym}
\usepackage{tikz}
\usetikzlibrary{arrows.meta,positioning,matrix}

\newcommand{\term}[1]{\texttt{#1}}
\newcommand{\li}{\!\multimap\!}
\newcommand{\lotimes}{\!\otimes\!}
\newcommand{\loplus}{\!\oplus\!}
\newcommand{\lin}[1]{\langle#1\rangle}
\newcommand{\lint}[1]{[#1]}

\title{Linear Logic}
\author{K. Kogkalidis}
\institute{Logic \& Language 2020}

\begin{document}
\date{}
\maketitle

\begin{frame}{Truth vs. Resource}
	\begin{block}{Linear Logic}
	Proposed by Girard (1987) as a resource-conscious alternative to classical \& intuitionistic logic.
	\end{block}
	\vfill
	
	\alt<3>{
	\begin{itemize}
		\item Propositions now represent \alert{resources}
		\item Resources are \alert{not} free to discard and replicate
		\item $\Rightarrow$ Contraction \& Weakening are not universally applicable\\
		\begin{flushright}
			\alert{Substructural!}
		\end{flushright}
		\item Inference rules can share contexts
	\end{itemize}}{
		\alt<2>{
		``
		\small
		\textit{Classical and intuitionistic logics deal with stable truths:
			\begin{center}
				if $A$ and $A\to B$, then $B$, but $A$ still holds.
			\end{center}
			This is perfect in mathematics, but wrong in real life, since real implication is causal.
			A causal implication cannot be iterated since the conditions are modified after its use; this process of modification of the premises is known in physics as reaction.
			For instance, if $A$ is to spend \$1 on a pack of a cigarettes and $B$ is to get them, you lose \$1 in this process, and you cannot do it a second time. 
			The reaction here was that \$1 went out of your pocket.}
		''
		}
		{}
	}
\end{frame}

\begin{frame}{Linear Logic: Syntax \& Connectives}
		Linear propositions $\mathcal{P}$ are defined as:
		\[
		\mathcal{P} := \mathcal{C}  \ | \ \mathcal{P}_1 \li \mathcal{P}_2 \ | \ \mathcal{P}_1 \lotimes \mathcal{P}_2 \
		| \ \mathcal{P}_1 \loplus \mathcal{P}_2 \ | \ !\mathcal{P}
		\]
	\vfill
	

	\begin{itemize}
		\item[$\li$] is read as ``lolli'' \\
			\begin{flushright}
			\small{$A\li B$: consume $A$ to produce a $B$}
			\end{flushright}		
		\item[$\lotimes$] is read as ``tensor''
			\begin{flushright}
			\small{$A\lotimes B$: both $A$ and $B$}
			\end{flushright}		
		\item[$\&$] is read as ``with''
			\begin{flushright}
			\small{$A\& B$: pick from $A$ and $B$}
			\end{flushright}
		\item[$\loplus$] is read as ``or''
			\begin{flushright}
			\small{$A\loplus B$: either $A$ and $B$}
			\end{flushright}
		\item[$!$] is read as ``bang''
			\begin{flushright}
			\small{$!A$: of course $A$}
			\end{flushright}
	\end{itemize}
\end{frame}

\begin{frame}{Universal Logic}
	\small	
	
	\begin{block}{Two kinds of resources}
		IL and LL can co-exist in peace:
		an \alert{assumption} $\mathcal{A}$ can be either \alert{linear} $\lin{\mathcal{A}}$ or \alert{intuitionistic} $\lint{\mathcal{A}}$; each comes with its own identity:
		\[
		\vcenter{\infer[\lin{Id}]{\lin{A} \vdash A}{}}  \quad \vcenter{\infer[\lint{Id}]{\lint{A} \vdash A}{}}
		\]
		\begin{flushright}
			$\Gamma, \Delta, \Theta, \dots$ will now denote \alert{sequences of assumptions}
		\end{flushright}
	\end{block}
	\pause
	
	\begin{block}{Intuitionistic Resources}
		permit contraction \& weakening:
		\[
			\vcenter{\infer[C]{\Gamma, \lint{A} \vdash B}{\Gamma, \lint{A}, \lint{A} \vdash B}}
			\quad\quad
			\vcenter{\infer[W]{\Gamma, \lint{A} \vdash B}{\Gamma \vdash B}}
		\]
		\pause
		and the introduction/elimination of !:
		\[
			\vcenter{\infer[!I]{\lint{\Gamma} \vdash !A}{\lint{\Gamma} \vdash A}}
			\quad\quad
			\vcenter{\infer[!E]{\Gamma, \Delta \vdash B}{
				\Gamma \vdash !A
				&
				\Delta, \lint{A} \vdash B
			}
			}
		\]
	\end{block}
\end{frame}

\begin{frame}{Logical Rules}
	\[
	\infer[\li I]{\Gamma \vdash A \li B}{\Gamma, \lin{A} \vdash B}
	\quad
    \infer[\li E]{\Gamma, \Delta \vdash B}
	{\Gamma \vdash A \li B
	&
	\Delta \vdash A}
	\]
	\pause

	\[
	\infer[\otimes I]{\Gamma, \Delta \vdash A \lotimes B}{
		\Gamma \vdash A
		&
		\Delta \vdash B
	}
	\quad
    \infer[\otimes E]{\Gamma, \Delta \vdash C }
	{\Gamma \vdash A \lotimes B
	&
	\Delta, \lin{A}, \lin{B} \vdash C}
	\]
	\pause
	
	\[
	\infer[\&I]{\Gamma \vdash A\&B}{
	\Gamma \vdash A
	& 
	\Gamma \vdash B
	}
	\quad
	\infer[\&E_1]{\Gamma \vdash A}{
		\Gamma \vdash A \& B
	}
	\quad
	\infer[\&E_2]{\Gamma \vdash B}{
		\Gamma \vdash A \& B
	}
	\]
	\pause
	
	\[
	\infer[\oplus E]{\Gamma, \Delta \vdash C}{
		\Gamma \vdash A\loplus B
		&
		\Delta, \lin{A} \vdash  C
		&
		\Delta, \lin{B} \vdash C
	}
	\]
	
	\[
	\infer[\oplus I_1]{\Gamma \vdash A \loplus B}{
		\Gamma \vdash A
	}
	\quad
	\infer[\oplus I_2]{\Gamma \vdash A \loplus B}{
		\Gamma \vdash B
	}
	\]
	\vfill
\end{frame}

\begin{frame}{Example}
	\small 
	\[	
		\infer[\visible<2->{!E}]{\lin{!(A\& B)} \vdash !A\otimes !B}{
		\visible<2->{
			\infer[\visible<3->{\lin{Id}}]{\lin{!(A\& B)} \vdash !(A\&B)}{}
			&
			\infer[\visible<4->{C}]{\lint{A\& B} \vdash !A \otimes !B}{
			\visible<4->{
				\infer[\visible<5->{\otimes I}]{\lint{A\& B}, \lint{A\& B} \vdash !A \otimes !B}{
				\visible<5->{
					\infer[\visible<6->{!I}]{\lint{A\& B} \vdash !A}{
					\visible<6->{
						\infer[\visible<7->{\&E_1}]{\lint{A\& B} \vdash A}{
						\visible<7->{
							\infer[\visible<8->{\lint{Id}}]{\lint{A\& B} \vdash A \& B}{
							}
						}}
					}}
					&
					\infer[\visible<9->{!I}]{\lint{A\& B} \vdash !B}{
					\visible<9->{
						\infer[\&E_2]{\lint{A\& B} \vdash B}{
							\infer[\lint{Id}]{\lint{A\& B} \vdash B}{}
						}
					}}
				}}
			}}
		}}
	\]
\end{frame}


\begin{frame}{Embedding}
	\begin{block}{IL in LL}
	Let $*$ an operator sending formulas of IL to formulas of ILL, such that:
			\begin{itemize}
				\item if $p \in \mathcal{A}$, then $p^* = p$
				\item otherwise:
				\begin{align*}
					(A \to B)^* &= \ !A^* \li B^*\\
					(A \times B)^* &= \ A^* \& B^*\\
					(A + B)^* &= \ !A^* \oplus !B^*
				\end{align*}
			\end{itemize}
	and judgements of IL to judgements of ILL:
		\[
			(\Gamma \vdash A)^* = \lint{\Gamma^*} \vdash A^*
		\]
	\end{block}
\end{frame}

\begin{frame}{Embedding Example}
	\small
	Proving the intuitionistic judgement $A, A\to B \vdash A\times B$ in ILL:
	\begin{align*}
		(A, A\to B \vdash A\times B)^* & = 
		\visible<2->{ \lint{A^*}, \lint{(A\to B)^*} \vdash (A\times B)^*}\\
		\visible<3->{& = \lint{A}, \lint{!A^*\li B^*} \vdash A^* \& B^*}\\
		\visible<4->{& = \lint{A}, \lint{!A \li B} \vdash A \& B}
	\end{align*}
	\vfill
	
	\visible<5->{
	\[
		\infer[\visible<6->{\& I}]{\lint{A}, \lint{!A\li B} \vdash A \& B}{
		\visible<6->{
			\infer[\visible<7->{W}]{\lint{A}, \lint{!A\to B}\vdash A}{
			\visible<7->{
				\infer[\lint{Id}]{\lint{A} \vdash A}{}
			}}
			&
			\infer[\visible<8->{Ex}]{\lint{A},\lint{!A\li B} \vdash B}{
			\visible<8->{
				\infer[\visible<9->{\li E}]{\lint{!A\li B}, \lint{A} \vdash B}{
				\visible<9->{
					\infer[\visible<10->{\lint{Id}}]{\lint{!A\li B}\vdash !A\li B}{}
					&
					\infer[\visible<11->{!I}]{\lint{A} \vdash !A}{
					\visible<11->{
						\infer[\lint{Id}]{\lint{A} \vdash A}{}
					}}
				}}
			}}
		}}
	\]
	}
\end{frame}

\begin{frame}{ILL $\stackrel{CH}{\equiv}$ Simply typed linear $\lambda$-calculus}
	\begin{block}{Linear $\lambda$-calculus}
		\begin{itemize}
			\item No vacuous abstractions: abstracted variables must be used in the function body
			\item All variables occur exactly once
		\end{itemize}
	\end{block}

\end{frame}

\begin{frame}{The simply typed linear $\lambda$-calculus}
	% no vacuous abstraction
	% exactly one occurrence in fun body
	\begin{align*}
		\mathcal{T} :=& \ 
					\visible<2->{\mathcal{V}}
					\visible<3->{\ | \ !\mathcal{T} \ | \ \term{case }\mathcal{T}\term{ of }!\mathcal{T}\to \mathcal{T}}
					\visible<4->{\ | \ \lambda\lin{\mathcal{T}}.\mathcal{T} \ | \ \mathcal{T}\lin{\mathcal{T}}}\\
					\visible<5->{& \ | \ \lin{\mathcal{T},\mathcal{T}} \ | \ \term{case }\mathcal{T}\term{ of }\lin{\mathcal{T},\mathcal{T}}\to \mathcal{T}}\\
					\visible<6->{& \ | \ \lin{\lin{\mathcal{T}, \mathcal{T}}}
					\ | \ \term{fst}\lin{\mathcal{T}} \ | \ \term{snd}{\lin{\mathcal{T}}}
					}\\
					\visible<7->{& \ | \ \term{inl}\lin{\mathcal{T}} \ | \ \term{inr}\lin{\mathcal{T}} \\ & \ | \ \term{case }\mathcal{T}\term{ of inl}\lin{\mathcal{T}}\to\mathcal{T};\term{inr}\lin{\mathcal{T}}\to \mathcal{T}}
	\end{align*}
	\vfill
	\small
	
	\alt<1>{}{
	\alt<2>{
		\[
		\vcenter{\infer[\lin{Id}]{\lin{\term{x}: A} \vdash\term{x}: A}{}}  
		\quad
		 \vcenter{\infer[\lint{Id}]{\lint{\term{x}: A} \vdash \term{x}:A}{}}
		\]
	}{
	\alt<3>{
		\[
			\vcenter{\infer[!I]{\lint{\Gamma} \vdash !\term{t}: !A}{\lint{\Gamma} \vdash \term{t}:A}}
			\quad
			\vcenter{\infer[!E]{\Gamma,\Delta \vdash \term{case s of }!x\to u: B}{
				\Gamma \vdash \term{s}: !A
				&
				\Delta, \lint{\term{x}: A} \vdash \term{u}: B
			}}
		\]
	}{
	\alt<4>{
		\[
			\vcenter{\infer[\li I]{\Gamma \vdash \lambda\term{x.y}: A\li B}{\Gamma, \lin{\term{x}: A} \vdash \term{y}: B}}
			\quad
			\vcenter{\infer[\li E]{\Gamma, \Delta \vdash \term{f}\lin{\term{x}}\footnote{Or: \term{f x}}: B}{
				\Gamma \vdash \term{f}: A\li B
				&
				\Delta \vdash \term{x}: A
			}}
		\]
	}{
	\alt<5>{
		\[
			\vcenter{\infer[\otimes I]{\Gamma \vdash \lin{\term{t}, \term{u}}: A\lotimes B}{
				\Gamma \vdash \term{t}: A
				&
				\Delta \vdash \term{u}: B
			}}
			\quad
			\vcenter{\infer[\otimes E]{\Gamma, \Delta \vdash \term{case s of }\lin{\term{x}, \term{y}}\to \term{v}: C}{
				\Gamma \vdash \term{s}: A \lotimes B
				&
				\Delta, \lin{\term{x}: A}, \lin{\term{y}: B} \vdash \term{v}: C
			}}
		\]
	}{
	\alt<6>{
		\[
			\vcenter{\infer[\& I]{\Gamma \vdash \lin{\lin{\term{t}, \term{u}}}: A\&B}{
				\Gamma \vdash \term{t}: A
				&
				\Gamma \vdash \term{u}: B
			}}
			\quad
			\vcenter{\infer[fst]{\Gamma \vdash \term{fst}\lin{\term{s}}: A}{
				\Gamma \vdash s: A\&B
			}}
			\quad
			\vcenter{\infer[snd]{\Gamma \vdash \term{snd}\lin{\term{s}}: B}{
				\Gamma \vdash s: A\&B
			}}
		\]
	}{
		\[
			\vcenter{\infer[\oplus I_1]{\Gamma \vdash \term{inl}\lin{\term{x}}: A\loplus B}{
				\Gamma \vdash \term{x}: A
			}}
			\quad
			\vcenter{\infer[\oplus I_2]{\Gamma \vdash \term{inr}\lin{\term{x}}: A\loplus B}{
				\Gamma \vdash \term{x}: B
			}}
		\]
		
		\[
			\vcenter{\infer[\oplus E]{\Gamma, \Delta \vdash \term{case s of inl}\lin{\term{x}}\to \term{u}; \term{inr}\lin{\term{y}}\to\term{w}: C}{
				\Gamma \vdash \term{s}: A\loplus B
				&
				\Delta, \lin{\term{x}: A} \vdash \term{u}: C
				&
				\Delta, \lin{\term{y}: B} \vdash \term{w}: C
			}}
		\]
	}}}}}}
\end{frame}

\begin{frame}{Proof Normalization \& Term Reduction ($!$)}
	\small 
	
	\[
		\begin{array}{ccc}
		\infer[\textcolor{red}{! E}]{\lint{\Gamma}, \Delta \vdash B}{
			\infer[\textcolor{red}{!I}]{\lint{\Gamma} \vdash \term{!t}: !A}{
				\infer*[]{\Gamma \vdash \term{t}: A}{}			
			}
			&
			\infer={\Gamma, \lint{\term{x}: A} \vdash \term{u}: B}{
				\infer*[]{\lint{\term{x}: A} \vdash \term{x}: A}{
					\lint{\term{x}: A} \vdash \term{x}: A
					&
					\dots
				}
			}
		}
		&
		\Longrightarrow
		&
		\infer={\lint{\Gamma}, \Delta \vdash \term{u}: B}{
			\infer*{\lint{\Gamma}, \dots, \Delta \vdash \term{u}: B}{
				\infer*{\lint{\Gamma} \vdash \term{t}: A}{}
			}
		}
		\end{array}
	\]	
	
	\alert{
	\[
		\term{case !t of !x}\to\term{u} \Longrightarrow \term{u[t/x]}
	\]
	}
\end{frame}

\begin{frame}{Proof Normalization \& Term Reduction ($\&$)}
	\small 
	
	\[
		\begin{array}[t]{@{}ccc}
			\infer[\textcolor{red}{\&E_1}]{\Gamma \vdash \term{fst}\lin{~\lin{\lin{\term{t,u}}}~}: A}{
				\infer[\textcolor{red}{\&I}]{\Gamma \vdash \lin{\lin{\term{t,u}}}: A\& B}{
				\infer*{\Gamma \vdash \term{t}: A}{}
				&
				\infer*{\Gamma \vdash \term{u}: B}{}
				}
			}		
			&
			\Rightarrow
			&
			\infer*{\Gamma \vdash \term{t}: A}{}
		\end{array}
	\]	
	
	\alert{
	\[
		\term{fst}\lin{~\lin{\lin{\term{t,u}}}~} \Longrightarrow \term{t}
	\]
	}
\end{frame}

\end{document}