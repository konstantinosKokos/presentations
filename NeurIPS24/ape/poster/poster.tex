\documentclass{article}
\usepackage[paperwidth=36in,paperheight=24in,hmargin=1.4in,vmargin=1in]{geometry}

\usepackage{graphicx}
\usepackage{tikz}
\tikzset{
    grid/.style={very thin,gray}
}
\tikzset{
    dot/.style={
        draw,
        gray!50,
         line width=1pt,
         minimum height=3.5pt,
         inner sep=0pt,
         anchor=center,
         outer sep=5pt
    },
    label/.style={text height =1.5ex}
}
\usepackage{tikz-qtree}
\usetikzlibrary{calc,positioning,arrows.meta,arrows}
\usepackage{enumitem}

\pagenumbering{gobble}


%\usepackage{xcolor}
%\definecolor{blue}{HTML}{46A5FF}
%\definecolor{yellow}{HTML}{F7E159}
%\definecolor{red}{HTML}{FD6360}


\usepackage[T1]{fontenc}
\usepackage{newpxtext,newpxmath}
\usepackage{eulervm}
\makeatletter
\def\txds@@scale{1}
\input{ot1tx-ds.fd}
\makeatother
\usepackage[bb=tx]{mathalpha}

\newcommand{\titlefont}{\fontsize{55}{55}\selectfont\setlength{\parskip}{1\baselineskip}}
\newcommand{\sectionfont}{\fontsize{34}{34}\selectfont\setlength{\parskip}{1\baselineskip}}
\newcommand{\nfont}{\fontsize{21}{22}\selectfont\setlength{\parskip}{1\baselineskip}}
\newcommand{\tikzfont}{\fontsize{10.5}{11}\selectfont\setlength{\parskip}{1.2\baselineskip}}

\newcommand{\light}[1]{\textcolor{gray!90}{#1}}

\newcommand{\Path}{\ensuremath{\mathbb{P}}}
\newcommand{\zero}{\ensuremath{\mathbb{0}}}
\newcommand{\one}{\ensuremath{\mathbb{1}}}
\newcommand{\two}{\ensuremath{\mathbb{2}}}
\newcommand{\pcomp}{\mathbin +}
\newcommand\pinvt[1]{\mathop{-}#1}
\newcommand\pinv[1]{(\pinvt{#1})}
\newcommand{\perm}[1]{\ensuremath{\mathcal{S}_{#1}}}
\newcommand\hmmax{0}
\newcommand\bmmax{0}
\newcommand{\inter}[1]{\ensuremath\lceil{#1}\rceil}


\newcommand{\ape}{{APE}}
\newcommand{\rope}{{RoPE}}
\newcommand{\ie}{\textit{i.e.},}
\newcommand{\ia}{\textit{i.a.},}

%\renewcommand{\quad}{\hphantom{x}}
\renewcommand{\section}[1]{\textbf{\sectionfont #1}}


\begin{document}
	% Title
	\begin{minipage}{\textwidth}
		\begin{minipage}{0.5\textwidth}
			\titlefont
			Algebraic Positional Encodings
		\end{minipage}%
		\begin{minipage}{0.3\textwidth}
			\nfont
			Konstantinos Kogkalidis\\
			Jean-Philippe Bernardy\\
			Vikas Garg
		\end{minipage}\hfill
		\begin{minipage}{0.15\textwidth}
			\includegraphics[width=\textwidth]{NeurIPS_logo.png}
		\end{minipage}
	\end{minipage}
	\vspace{0.5in}
	
	% Columns
	\begin{minipage}[t]{0.29\textwidth}
		\sectionfont
		\textbf{TL;DR}
		
		\nfont	
		\hfill\rule[0\baselineskip]{0.65\textwidth}{1pt}
		
		\hfill\begin{minipage}{0.65\textwidth}
		``\textit{%
					syntax is an algebra,\\
					semantics is an algebra,\\
					and meaning is a homomorphism between them%
				}''\\
		\end{minipage}
		\vspace{-\baselineskip}
		\hfill\rule{.65\textwidth}{1pt}
		
		\hfill Montague's theory of meaning
		
		We argue that:
		\begin{itemize}[topsep=-1\baselineskip,noitemsep]
			\item understanding and explicating the formation rules and rewrite properties of \textbf{positions} over different \textbf{ambient structures} (\textit{syntax})
			\item and finding appropriate structure-preserving \textbf{interpretations} (\textit{meaning})
		\end{itemize}
		is the only way to structure-faithful \textbf{positional encodings} (\textit{semantics}).
		
		We call these Algebraic Positional Encodings (\ape). APE readily apply to:
		\begin{itemize}[topsep=-1\baselineskip,noitemsep]
			\item sequences
			\item trees
			\item grids
			\item ...
		\end{itemize}\vspace{\baselineskip}

		We show that \textbf{sequential \ape{} theoretically subsume \rope}. 	Beyond sequences, \ape{} are a \textbf{theoretically disciplined and highly general extension of \rope{} across multiple dimensions} \light{(both metaphorical and literal)}.
		
		\sectionfont
		\vspace{\parskip}
		\textbf{Sequences}
		
		\nfont
		Let $\Path$ be a \textit{path} (\ie{} a relative offset) between two points in a sequence.
		
		$\Path$ admits a simple inductive definition:
		\begin{align*}
		\Path & :=  \one & & \light{\text{\# take a step to the right}}\\
		      & \quad | ~ \Path + \Path & & \light{\text{\# join two paths together}}\\
		      & \quad | ~ \Path^{-1} & & \light{\text{\# flip a path around}}
		\end{align*}
		where $+$ associative and commutative with $\zero := \one + \one^{-1}$ as its neutral element.
		
		\textbf{Remark 1.} The signature coincides with that of the integers, $\Path \equiv \mathbb{Z}$.\\
		\textbf{Remark 2.} The signature corresponds to an infinite cyclic group, $\Path \equiv \langle \one \rangle$.\\
		\textbf{Remark 3.} The signature admits a representation in $O(n)$.\\
		\hphantom{x}Consider the interpretation $\inter{}: \langle \one \rangle \to \langle W \rangle$, such that:
		\begin{align*}
			\inter{\one} & \mapsto W & & \light{\text{\# $W$ represents a single step}}\\
			\inter{p + q} & \mapsto \inter{p} \inter{q} & & \light{\text{\# path composition $\leadsto$ matrix multiplication}}\\
			\inter{p^{-1}} & \mapsto \inter{p}^{-1} & & \light{\text{\# path inversion $\leadsto$ matrix transposition}}
		\end{align*}
		\textbf{Remark 4.} $A \to B = (A \to \zero) + (\zero \to B)$. Visually:
		\begin{center}
		\begin{tikzpicture}[scale=2, transform shape]\tikzfont
	        \draw[grid] (0,0) -- (4.5,0);
	        \draw[grid,dashed] (4.5,0) -- (5.5,0);
	        \node[dot] (0) at (0,0) {};
	        \node[below=-5pt of 0,label] {$\zero$};
	        \node[dot] (1) at (1,0) {};
	        \node[below=-5pt of 1,label] {$\one$};
	        \node[dot] (2) at (2,0) {};
	        \node[below=-5pt of 2,label] {$\one^{2}$};
	        \node[dot] (3) at (3,0) {};
	        \node[below=-5pt of 3,label] {$\one^{3}$};
	        \node[dot] (4) at (4,0) {}; 
	        \node[below=-5pt of 4,label] {$\one^{4}$};
	        
	        \draw[->,dotted,very thick] (1) to[bend left=45] node[midway, above]{${W}^{-1}{W}^4 = {W}^{3}$} (4);
	        \draw[<-,dotted,very thick] (1) to[bend right=45] node[midway, below]{${W}^{-4}{W} = {W}^{-3}$} (4);
	    \end{tikzpicture}
		\end{center}
		\textbf{Remark 5.} This setup offers an inductive parameterization of sequential PE using just \textbf{one trainable primitive} (a single matrix).
				
	\end{minipage}\hfill
	\begin{minipage}[t]{0.28\textwidth}
		\sectionfont	
		\textbf{How-To}
		
		\nfont
		
		Simply substitute dot-product for the tensor contraction:
		\begin{center}
		\begin{tikzpicture}[scale=2, transform shape]\tikzfont
			\tikzstyle{v} = [
			    draw, 
			    ultra thick, 
			    fill=blue!20, 
			    circle, 
			    rounded corners, 
%			    text centered, 
			]
			\tikzstyle{m} = [
			    draw, 
			    ultra thick, 
			    fill=green!20, 
			    rectangle, 
			    rounded corners, 
%			    text centered, 
			]

			\node[v] at (0, 0) (q) {$q$};
			\node[m] at (2, 0) (fq) {$\Phi^{(q)}$};
			\draw[ultra thick] (q) -- (fq) node[above, midway] {$\alpha$};
			
			\node[m] at (5, 0) (aq) {$A^{(q)}$};
			\draw[ultra thick] (fq.east)
			    .. controls +(0.5,0) and +(-0.5,0.0) .. ++(1, 0.5) node[midway,above] {$\beta$}
			    -- ++(2, 0)
			    .. controls +(0.35,0) and +(0.35,0) .. 
			    	(aq.east);

			\node[v] at (0, -2) (k) {$k$};			
			\node[m] at (2, -2) (fk) {$\Phi^{(k)}$};
			\node[m] at (5, -2) (ak) {$A^{(k)}$};
			\draw[ultra thick] (k) -- (fk) node[below, midway] {$\epsilon$};

			\draw[ultra thick] (aq.west)
				.. controls +(-0.35, 0) and +(-0.35, 0) .. ++(0, -0.5)
				-- ++ (+0.5, 0)
				.. controls +(1.15, 0) .. ++(1.15, -0.5)
				-- +(0, -1) node[midway,right] {$\gamma$}
				.. controls +(-0, -0.5) .. ++ (-0.65, -1.5)
				-- ++(-1, 0)
				.. controls +(-0.35, 0) and +(-0.35, 0) .. +(0, +0.5)
			;

			\draw[ultra thick] (fk.east)
			    .. controls +(0.5,0) and +(-0.5,0.0) .. ++(1, 0.5) node[midway,above] {$\delta$}
			    -- ++(2, 0)
			    .. controls +(0.35,0) and +(0.35,0) .. 
			    	(ak.east);
			    	
			\draw[dashed, ultra thick, gray, rounded corners] (3.5, 1) -- (6.85, 1) -- (6.85, -3) -- (3.5, -3) -- cycle;
			\node[anchor=west] at (7, 1) {$T^{(q\to k)}$};
		\end{tikzpicture}
		\end{center}
		where:
		\begin{itemize}[noitemsep,topsep=-\baselineskip]
			\item $q, k \in \mathbb{R}^{n}$
			\item $\Phi^{(q,k)} \in \mathbb{R}^{n\times n}$
			\item $A^{(q,k)} \in O(n)$ the representations of the positions of $q$ and $k$\\
			\light{\hfill \textbf{Note}: $T^{(q\to k)} = {A^{(q)}}^\top A^{(k)}$ the \textbf{path} representation from $q$ to $k$}
		\end{itemize}
		\vspace{\parskip}
	
		In the sequential setup RoPE $\equiv$ APE, except with a fixed $W$. Why? \\
		\vspace{-2\baselineskip}
		\begin{flushright}
		\light{
			\textbf{Hint}: $W = QRQ^\top$ (where $Q \in O(n)$ and $R$ a block-diagonal rotation).
		}
		\end{flushright}		 
		
		\sectionfont
		\vspace{\parskip}
		\textbf{Trees}
		
		\nfont
		Extend the definition of $\Path$ with \textbf{options}, to arrive at a definition of paths $\Path_{\kappa}$ over $\kappa$-ary branching trees:
		\begin{align*}
			\Path_{\kappa} & := 
			 \one & & \light{\text{\# take the first branch}}\\
			  & \quad | ~ \two & & \light{\text{\# take the second branch}}\\
			  & \quad | ~ \dots\\
 			  & \quad | ~ \kappa & & \light{\text{\# take the $\kappa$-th branch}}\\
		      & \quad | ~ \Path + \Path & & \light{\text{\# join two paths together}}\\
		      & \quad | ~ \Path^{-1} & & \light{\text{\# flip a path around}}
		\end{align*}
		
		\textbf{Remark 5.} This is now a generic group with $\kappa$ generators.\\
		\textbf{Remark 6.} Unlike sequences, the structure is not commutative.\\
		\textbf{Remark 7.} All else remains the same -- just extend the interpretation to:\\
		$ \langle \one,\two,\dots,\kappa \rangle \to \langle W_1, W_2,\dots,W_\kappa \rangle$. Visually:

		
		\begin{center}
		\begin{tikzpicture}[grow'=up,anchor=center,sibling distance=0.8cm,level distance=1.33cm,scale=2, transform shape]\tikzfont
		    \tikzset{level 3/.style={sibling distance=13pt}}
		    \tikzset{level 3/.style={level distance=30pt}}
		    \tikzset{edge from parent/.style={grid,draw}}
		    \tikzset{every tree node/.style={label}}
		            \Tree 
		            [.\node (1) {$\zero$};
		                [.\node (2) {$\one$};
		                    [.\node (4) {$\one\one$};
		                        \edge[dashed]; {}
		                        \edge[dashed]; {}
		                    ]
		                    [.\node (5) {$\one\two$};
		                        \edge[dashed]; {}
		                        \edge[dashed]; {}
		                    ]
		                ]
		                [.\node (3) {$\two$};
		                    [.\node(6) {$\two\one$};
		                        \edge[dashed]; {}
		                        \edge[dashed]; {}
		                    ]
		                    [.\node(7) {$\two\two$};
		                        \edge[dashed]; {}
		                        \edge[dashed]; {}
		                    ]
		                ]
		            ]  
		        \draw [->,dotted,very thick,rounded corners,transform canvas={shift={(0,5pt)}}]
		            ($(6.south) + (7pt, -3pt)$) --
		            ($(3.north) + (-5pt,6pt)$) -- 
		            ($(3.south) + (-5pt,-2pt)$) -- 
		            ($(1.north) + (0, 0.8pt)$) -- 
		            ($(2.south) + (5pt,-2pt)$) -- 
		            ($(2.north) + (5pt,6pt)$) -- 
		            ($(5.south) + (-7pt,-3pt)$);
		        \node (x) at (0pt, 41pt) {$(W_2W_1)^{-1}W_{1}W_2$};
		    \end{tikzpicture}
		    \sectionfont
			\vspace{\parskip}
				    
%			\includegraphics[width=0.3\textwidth]{qr.png}		    
		    
		\end{center}
	\end{minipage}\hfill
	\begin{minipage}[t]{0.286\textwidth}
		\sectionfont
		\textbf{Grids}
		\nfont
		
		Rather than add options, we can glue two (or more) sequences together by means of the \textbf{group direct sum}, $\oplus$.
		Consider the composite group $\Path^2 := \Path \oplus \Path$, with the group operation and inversion defined as:
		\begin{align*}
			(x, y) + (z,w) &= (x+z,y+w)\\
			(x, y)^{-1} &= (x^{-1}, y^{-1})
		\end{align*}
		
		\textbf{Remark 8.} The structure is commutative once more.\\
		\textbf{Remark 9.} Elements of $\Path^2$ are still to be interpreted as (orthogonal) matrices, except now block-structured, by virtue of the \textbf{matrix direct sum}:
		\begin{align*}
			\inter{p \oplus q} \mapsto \inter{p} \oplus \inter{q} = \begin{bmatrix}
			\inter{p} & 0\\
			0 & \inter{q}
			\end{bmatrix}
		\end{align*}
		Visually:\\
		\vspace{-\baselineskip}
		\begin{center}
			\begin{tikzpicture}[scale=2, transform shape]\tikzfont
		        \draw[step=1cm,grid] (0,0) grid (2.99,2.99);
		        \draw[dashed,grid] (0, 3) -- (3, 3);
		        \draw[dashed,grid] (3, 0) -- (3, 3);
		        \draw[dashed,grid] (0, 3) -- (0, 4);
		        \draw[dashed,grid] (1, 3) -- (1, 4);
		        \draw[dashed,grid] (2, 3) -- (2, 4);
		        \draw[dashed,grid] (3, 3) -- (3, 4);
		        \draw[dashed,grid] (3, 0) -- (4,0);f
		        \draw[dashed,grid] (3, 1) -- (4,1);
		        \draw[dashed,grid] (3, 2) -- (4,2);
		        \draw[dashed,grid] (3, 3) -- (4,3);
		
		        \foreach \x in {0,1,2,3} {
		        \foreach \y in {0,1,2,3} {
		            \filldraw[gray!90] (\x,\y) circle (1pt);
		           }
		        }
		        \node[left, label,fill opacity=0.5, text opacity=1.] at (0, 3) {$(\one^3, \zero)$};
		        \node[right, label,fill opacity=0.5, fill=white, text opacity=1.] at (3,1) {$(\one, \one^3)$};
		        \draw[->, dotted, very thick]
		            (0.15, 3) 
		            to[bend left=40] 
		            node[midway,above,right,fill opacity=0.6, fill=white, text opacity=1.,xshift={0.25cm}]{${H}^{-2}\oplus{W}^3$} 
		            (2.95, 1.15);
		        \draw[<-, dotted, very thick] 
		            (0, 2.85) 
		            to[bend right=40] 
		            node[midway,below,left,fill opacity=0.4, fill=white, text opacity=1., xshift={-0.25cm}]{${H}^{2}\oplus{W}^{-3}$} 
		            (2.85,1);
		    \end{tikzpicture}
		\end{center}
		\textbf{Remark 10.} The same interpretation strategy can be applied to construct \textbf{any other composition} of established structures and their representations.
		
		\sectionfont
		\vspace{\parskip}
		\textbf{Read More}
		\nfont
		
		Read the paper for:
		\begin{itemize}[noitemsep,topsep=-\baselineskip]
			\item prose
			\item tables with numbers
			\item links to code
			\item references
			\item ... other things papers usually have
		\end{itemize}
		
		\vspace{\parskip}\vspace{\parskip}\vspace{\parskip}\vspace{\parskip}
		\centering\includegraphics[width=0.25\textwidth]{qr.png}
	\end{minipage}%

\end{document}

